%% abtex2-modelo-trabalho-academico.tex, v-1.7.1 laurocesar
%% Copyright 2012-2013 by abnTeX2 group at http://abntex2.googlecode.com/ 
%%
%% This work may be distributed and/or modified under the
%% conditions of the LaTeX Project Public License, either version 1.3
%% of this license or (at your option) any later version.
%% The latest version of this license is in
%%   http://www.latex-project.org/lppl.txt
%% and version 1.3 or later is part of all distributions of LaTeX
%% version 2005/12/01 or later.
%%
%% This work has the LPPL maintenance status `maintained'.
%% 
%% The Current Maintainer of this work is the abnTeX2 team, led
%% by Lauro César Araujo. Further information are available on 
%% http://abntex2.googlecode.com/
%%
%% This work consists of the files abntex2-modelo-trabalho-academico.tex,
%% abntex2-modelo-include-comandos and abntex2-modelo-references.bib
%%

% ------------------------------------------------------------------------
% ------------------------------------------------------------------------
% abnTeX2: Modelo de Trabalho Academico (tese de doutorado, dissertacao de
% mestrado e trabalhos monograficos em geral) em conformidade com 
% ABNT NBR 14724:2011: Informacao e documentacao - Trabalhos academicos -
% Apresentacao
% ------------------------------------------------------------------------
% ------------------------------------------------------------------------
%
% ------------------------------------------------------------------------
% ------------------------------------------------------------------------
% unifeso-abnTeX2: Modelo de Trabalho Academico do curso de Ciência da
% Computação da Unifeso
% Derivado da classe ABNTex, em conformidade com ABNT NBR 14724:2011
% Informacao e documentacao - Trabalhos academicos - Apresentacao
% ------------------------------------------------------------------------
% ------------------------------------------------------------------------


\documentclass[
	% -- opções da classe memoir --
	12pt,				% tamanho da fonte
	openright,			% capítulos começam em pág ímpar (insere página vazia caso preciso)
	oneside,			% para impressão em verso e anverso. Oposto a oneside
	a4paper,			% tamanho do papel. 
	% -- opções da classe abntex2 --
	%chapter=TITLE,		% títulos de capítulos convertidos em letras maiúsculas
	%section=TITLE,		% títulos de seções convertidos em letras maiúsculas
	%subsection=TITLE,	% títulos de subseções convertidos em letras maiúsculas
	%subsubsection=TITLE,% títulos de subsubseções convertidos em letras maiúsculas
	% -- opções do pacote babel --
	english,			% idioma adicional para hifenização
	french,				% idioma adicional para hifenização
	spanish,			% idioma adicional para hifenização
	brazil,				% o último idioma é o principal do documento
	]{unifeso-abntex2}


% ---
% PACOTES
% ---

% ---
% Pacotes fundamentais 
% ---
\usepackage{cmap}				% Mapear caracteres especiais no PDF
% \usepackage{lmodern}			% Usa a fonte Latin Modern
\usepackage{mathptmx}			% Usa a fonte Times New Roman
\usepackage[T1]{fontenc}		% Selecao de codigos de fonte.
\usepackage[utf8]{inputenc}		% Codificacao do documento (conversão automática dos acentos)
\usepackage{lastpage}			% Usado pela Ficha catalográfica
\usepackage{indentfirst}		% Indenta o primeiro parágrafo de cada seção.
\usepackage{color}				% Controle das cores
\usepackage{graphicx}			% Inclusão de gráficos
\usepackage{lipsum}             % Pacote para Lorem Ipsum.
\usepackage{float}              % Pacote para posicionamento de elementos, e.g. Tabelas, figuras, equações, etc.
% ---
		

\usepackage{float}
\usepackage{subfig}


% ---
% Pacotes de citações
% ---
\usepackage[brazilian,hyperpageref]{backref}	 % Paginas com as citações na bibl
\usepackage[num]{abntex2cite}	% Citações padrão ABNT

%-------------------
%COMANDO PARA NOTAS DE RODAPE
\usepackage{hyperref}
\newcommand\fnurl[2]{%
\href{#2}{#1}\footnote{\url{#2}}%
}
%

% --- 
% CONFIGURAÇÕES DE PACOTES
% --- 

% ---
% Configurações do pacote backref
% Usado sem a opção hyperpageref de backref
\renewcommand{\backrefpagesname}{Citado na(s) página(s):~}
% Texto padrão antes do número das páginas
\renewcommand{\backref}{}
% Define os textos da citação
\renewcommand*{\backrefalt}[4]{
	\ifcase #1 %
		Nenhuma citação no texto.%
	\or
		Citado na página #2.%
	\else
		Citado #1 vezes nas páginas #2.%
	\fi}%
% ---


% ---------------------------------------------------------------------------------------------
% ---------------------------------------------Informações de dados para CAPA e FOLHA DE ROSTO-
% ---------------------------------------------------------------------------------------------
\titulo{PhotoGraph: Modelagem de Fluxos de Edição Fotográfica Através de Estruturas de Grafos}
\autor{Matheus da Silva Correa Motizuki}
\local{Teresópolis}
\data{\the\year}
\orientador{Victor de Almeida Thomaz}
%\coorientador{Nome do Co-orientador} % se não existir, comente essa linha
\instituicao{%
  Centro Universitário Serra dos Órgãos - UNIFESO
  \par
  Bacharelado em Ciência da Computação}
\tipotrabalho{Monografia (Graduação)}
% O preambulo deve conter o tipo do trabalho, o objetivo, 
% o nome da instituição e a área de concentração 
\preambulo{Trabalho de Conclusão de Curso apresentado ao Centro Universitário Serra dos Órgãos como requisito obrigatório para obtenção do título de Bacharel em Ciência da Computação.}
% ---


% ---
% Configurações de aparência do PDF final

% alterando o aspecto da cor azul
\definecolor{blue}{RGB}{41,5,195}

% informações do PDF
\makeatletter
\hypersetup{
     	%pagebackref=true,
		pdftitle={\@title}, 
		pdfauthor={\@author},
    	pdfsubject={\imprimirpreambulo},
	    pdfcreator={LaTeX with abnTeX2},
		pdfkeywords={UNIFESO}{ccomp}{unifeso-abntex2}{trabalho acadêmico}, 
		colorlinks=true,       		% false: boxed links; true: colored links
    	linkcolor=black,          	% color of internal links
    	citecolor=blue,        		% color of links to bibliography
    	filecolor=magenta,      		% color of file links
		urlcolor=blue,
		bookmarksdepth=4
}
\makeatother
% --- 

% --- 
% Espaçamentos entre linhas e parágrafos 
% --- 
% O tamanho do parágrafo é dado por:
\setlength{\parindent}{1.3cm}
% Controle do espaçamento entre um parágrafo e outro:
\setlength{\parskip}{0.2cm}  % tente também \onelineskip

\usepackage{tikz}
\usetikzlibrary{shapes,arrows}
\usepackage{amsmath}
\tikzstyle{block} = [rectangle, draw, fill=blue!20, 
    text width=5em, text centered, rounded corners, minimum height=4em]
\tikzstyle{line} = [draw, -latex']

\usepackage{listings}
\definecolor{dkgreen}{rgb}{0,0.6,0}
\definecolor{gray}{rgb}{0.5,0.5,0.5}
\definecolor{mauve}{rgb}{0.58,0,0.82}
 
\lstset{
  language=Python,                
  basicstyle=\footnotesize,           
  numbers=left,                   
  numberstyle=\tiny\color{gray},  
  stepnumber=2,                             
  numbersep=5pt,                  
  backgroundcolor=\color{white},    
  showspaces=false,               
  showstringspaces=false,         
  showtabs=false,                 
  frame=single,                   
  rulecolor=\color{black},        
  tabsize=2,                      
  captionpos=b,                   
  breaklines=true,                
  breakatwhitespace=false,        
  title=\lstname,                               
  keywordstyle=\color{blue},          
  commentstyle=\color{dkgreen},       
  stringstyle=\color{mauve},     
}



% ---
% compila o indice
% ---
\makeindex

% -----------------------------------------------------------------------------------
% -----------------------------------------------------------------------------------
% -----------------------------------------------------------------------------------
% ---------------------------------------------------------------Início do documento-
% -----------------------------------------------------------------------------------
% -----------------------------------------------------------------------------------
% -----------------------------------------------------------------------------------
\begin{document}

% Retira espaço extra obsoleto entre as frases.
\frenchspacing 

% -----------------------------------------------------------------------------------
% ------------------------------------------------------------ELEMENTOS PRÉ-TEXTUAIS-
% -------------------------------------------Altere o arquivo 00_pretextual.tex para-
% ---------------folha de aprovação, dedicatória, agradecimento e epígrafe e resumos-
% -----------------------------------------------------------------------------------
\pretextual

% ---------------------------------------------------------------------------------------------
% ----------------------------------------------------------------------------------------Capa-
% ---------------------------------------------------------------------------------------------
\imprimircapa
% ---

% ---------------------------------------------------------------------------------------------
% ------------------------------------------------------------------------------Folha de rosto-
% ---------------------------------------------------------------------------------------------
\imprimirfolhaderosto

% ---------------------------------------------------------------------------------------------
% --------------------------------------------------------------------------Folha de aprovação-
% -------------------------------------------------------Dedicatória, Agradecimento e Epígrafe-
% --------------------------------------------------------------Resumos, em português e inglês-
% ---------------------------------------------------------------------------------------------
% ---
% Inserir a ficha bibliografica
% ---
% A biblioteca da universidade lhe fornecerá um PDF
% com a ficha catalográfica definitiva após a defesa do trabalho. Quando estiver
% com o documento, salve-o como PDF no diretório do seu projeto e substitua todo
% o conteúdo de implementação deste arquivo pelo comando abaixo:

\begin{fichacatalografica}
    %\includepdf{fichacatalografica.pdf}
\end{fichacatalografica}


\newpage
% ---
% ---


% Isto é um exemplo de Folha de aprovação, elemento obrigatório da NBR
% 14724/2011 (seção 4.2.1.3). Você pode utilizar este modelo até a aprovação
% do trabalho. Após isso, substitua todo o conteúdo deste arquivo por uma
% imagem da página assinada pela banca com o comando abaixo:
%
% \includepdf{folhadeaprovacao_final.pdf}
%
\begin{folhadeaprovacao}

  \begin{center}
    {\ABNTEXchapterfont\large\imprimirautor}

    \vspace*{\fill}\vspace*{\fill}
    {\ABNTEXchapterfont\bfseries\Large\imprimirtitulo}
    \vspace*{\fill}
    
    \hspace{.45\textwidth}
    \begin{minipage}{.5\textwidth}
        \imprimirpreambulo
    \end{minipage}%
    \vspace*{\fill}
   \end{center}
 

    %\todo[inline]{Alterar o texto, talvez criar alguma macro condicional}
    \begin{center}
        \noindent
        Trabalho de Conclusão de Curso apresentado ao Centro Universitário Serra dos Órgãos como requisito obrigatório para obtenção do título de Bacharel em Ciência da Computação.
    \end{center}
   % Caso prefira fixar a data, basta mudar o comando \today pela data,
   %   como no exemplo abaixo:
   %Trabalho aprovado. \imprimirlocal, 7 de setembro de 1822:

    
   \assinatura{\textbf{\imprimirorientador} \\ Orientador} 
   \assinatura{\textbf{Professor} \\ Nome Membro 1}     % título e nome
   \assinatura{\textbf{Doutor} \\ Nome membro 2}        % título e nome
   %\assinatura{\textbf{Professor} \\ Convidado 3}
   %\assinatura{\textbf{Professor} \\ Convidado 4}
      
   \begin{center}
    \vspace*{0.5cm}
    {\large\imprimirlocal}
    \par
    {\large\imprimirdata}
    \vspace*{1cm}
  \end{center}
  
\end{folhadeaprovacao}

% ---------------------------------------------------------------------------------------------
% ---------------------------------------------------------------------------------Dedicatória-
% ---------------------------------------------------------------------------------------------
\begin{dedicatoria}
	\vspace*{\fill}
        \flushright
	% \centering
	% \noindent
	\textit{Dedico este trabalho ao meu falecido pai,\\
    que sempre me apoiou.}
\end{dedicatoria}

% ---------------------------------------------------------------------------------------------
% ------------------------------------------------------------------------------Agradecimentos-
% ---------------------------------------------------------------------------------------------
\begin{agradecimentos}
    Agradeço a Deus pela força e discernimento que me permitiram concluir esta etapa acadêmica. Agradeço a minha namorada por me acompanhar nessa jornada, sempre ao meu lado. Agradeço aos meus pais, que me apoiaram durante esta jornada acadêmica. Ao professor Victor de Almeida Thomaz, minha gratidão pela orientação e compartilhamento de conhecimentos essenciais para este trabalho.
\end{agradecimentos}

% ---------------------------------------------------------------------------------------------
% ------------------------------------------------------------------------------------Epígrafe-
% ---------------------------------------------------------------------------------------------
\begin{epigrafe}
	\vspace*{\fill}
	\begin{flushright}
	\textit{``Não vos amoldeis às estruturas deste mundo, \\
			mas transformai-vos pela renovação da mente, \\
			a fim de distinguir qual é a vontade de Deus: \\
			o que é bom, o que Lhe é agradável, o que é perfeito.\\
			(Bíblia Sagrada, Romanos 12, 2)} % Esse é um exemplo de epígrafe.
	\end{flushright}
\end{epigrafe}
% ---

% ---------------------------------------------------------------------------------------------
% -------------------------------------------------------------------------resumo em português-
% ---------------------------------------------------------------------------------------------
\begin{resumo}
	\vspace{\onelineskip}

Aplicativos de edição de imagens existem há muito tempo e, desde a sua criação, não deixam de apresentar desafios de usabilidade para usuários sem conhecimento técnico. O objetivo deste trabalho é desenvolver um aplicativo de edição de imagens simplificado e mais direto, utilizando o paradigma de programação de fluxo de dados, onde dados fluem entre nós através por arestas que estão interconectadas em um grafo direcionado, para isso, três soluções de software distintas foram criadas, a fim de comparar diferentes linguagens de programação e suas capacidades na implementação do paradigma de programação de fluxo de dados. A criação de tal aplicativo traz vantagens significativas, como a redução da curva de aprendizado para novos usuários e a possibilidade de indivíduos sem experiência técnica editarem imagens de forma eficaz. Além disso, o uso do paradigma de programação de fluxo de dados proporciona uma abordagem modular e flexível para a edição de imagens, facilitando a experimentação e criatividade dos usuários. Criando assim novas possibilidades para o campo de edição de imagens, democratizando o acesso a ferramentas avançadas.

\noindent 

\textbf{Palavras-chave}: Programação de fluxo de dados, Grafo Direcionado, Edição de imagens, Usabilidade, Praticidade.

\end{resumo}

% ---------------------------------------------------------------------------------------------
% ----------------------------------------------------------------------------resumo em inglês-
% ---------------------------------------------------------------------------------------------
\begin{resumo}[Abstract]
	\begin{otherlanguage*}{english}
    \vspace{\onelineskip}
    Image editing tools have existed for a long time and, since their inception, these software solutions have always presented usability challenges for non-technical users. The aim of this work is to develop a simplified and more straightforward image editing application, specifically designed for end users without technical expertise. As a key differentiator, the project explores the use of visual technologies, employing visual programming based on directed graphs to make the editing process more intuitive and accessible.

    \noindent
\textbf{Keywords}: Visual Programming, Image Editor, Directed Graph.
\end{otherlanguage*}
\end{resumo}

% ---------------------------------------------------------------------------------------------
% --------------------------------------------------------------------inserir lista de figuras-
% --------------------------------------------se não houver nenhuma figura, comente esta seção-
% ---------------------------------------------------------------------------------------------
\pdfbookmark[0]{\listfigurename}{lof}
\listoffigures*
\cleardoublepage
% ---

% ---------------------------------------------------------------------------------------------
% --------------------------------------------------------------------inserir lista de tabelas-
% --------------------------------------------se não houver nenhuma tabela, comente esta seção-
% ---------------------------------------------------------------------------------------------
 % \pdfbookmark[0]{\listtablename}{lot}
 % \listoftables*
 % \cleardoublepage
% ---

% ---------------------------------------------------------------------------------------------
% ------------------------------------------------------inserir lista de abreviaturas e siglas-
% ---------------------------------------------se não houver nenhuma sigla, comente esta seção-
% ---------------------------------------------------------------------------------------------
\begin{siglas}
        \item[PFD]  Programação de fluxo de dados
        \item[DFP]  Dataflow programming
        \item[CSS]  Cascading Style Sheets
\end{siglas}
% ---

% ---------------------------------------------------------------------------------------------
% -------------------------------------------------------------------inserir lista de símbolos-
% --------------------------------------------------se não houver símbolos, comente esta seção-
% ---------------------------------------------------------------------------------------------
 % \begin{simbolos}
 %   \item[$ \Gamma $] Letra grega Gama
 %   \item[$ \Lambda $] Lambda
 %   \item[$ \zeta $] Letra grega minúscula zeta
 %   \item[$ \in $] Pertence
 % \end{simbolos}
% ---

% ---------------------------------------------------------------------------------------------
% -------------------------------------------------------------------------------------Sumário-
% ---------------------------------------------------------------------------------------------
\pdfbookmark[0]{\contentsname}{toc}
\tableofcontents*
\cleardoublepage
% ---


% ---------------------------------------------------------------------------------------------
% ---------------------------------------------------------------------------------------------
% --------------------------------------------------------------------------ELEMENTOS TEXTUAIS-
% --------------------------Como exemplo, foram inseridos dois capítulos em arquivos separados-
% -------------------------Com o \include, pode-se inserir quantos capítulos forem necessários-
% -----------------------------------Arquivos separados auxiliam na organização final do texto-
% ---------------------------------------------------------------------------------------------
% ---------------------------------------------------------------------------------------------
\textual

%Capítulo 0X - Introdução
\chapter[Introdução]{Introdução}

% Introdução !!!!

Os aplicativos de edição de imagens têm uma longa trajetória. Um dos marcos iniciais que viria a se tornar o Adobe Photoshop, criado pelos irmãos John e Thomas Knoll em 1987. Após um período de testes e pequenas distribuições sob o nome de Barneyscan XP, em 1989, o programa chamou a atenção da Adobe Systems, que fez um contrato de licenciamento com os irmãos Knoll. Assim, em 1990, foi lançado oficialmente o Adobe Photoshop 1.0, inicialmente para o sistema Macintosh \cite{BarneyScan}.

Com o avanço das tecnologias de hardware e software, surgiram novos programas de edição de imagem, estabelecendo padrões que hoje são referências na indústria do design gráfico. Ferramentas como o próprio Photoshop, que desde a sua concepção vêm se adaptando aos padrões de cada época, e o Gimp (GNU Image Manipulation Program) tornaram-se amplamente reconhecidos \cite{PhotoEditors}.

Ao longo do tempo, os softwares de edição de imagem buscaram tornar suas interfaces mais intuitivas e simplificadas, a fim de atingir um público mais amplo e não técnico. Essa tendência de design, proporcionou a criação de diversos aplicativos, dentro e fora da área de edição de imagens, mais simples e com um menor requerimento de expertise em uma determinada área.

Um conceito que se destacou durante esse processo foi o desenvolvimento de interfaces visuais, ou programação visual. Isso ocorre porque, de maneira geral, a compreensão de informações apresentadas graficamente tendem a ser mais intuitivas. Essa abordagem reforça a importância de interfaces gráficas bem planejadas, permitindo com que os usuários apliquem seus conhecimentos de forma mais prática, otimizando o fluxo de trabalho e a produtividade.

A Programação de Fluxo de Dados (PFD), ou \textit{Dataflow Programming (DFP)}, é um paradigma de programação que modela a lógica de um programa como um grafo dirigido, de forma semelhante aos Diagramas de Fluxo de Dados (DFD). Nesse modelo, o programa é representado como um conjunto de nós, também conhecido como vértices interligados por conexões direcionadas, chamadas de arcos (ou bordas), que indicam o percurso dos dados entre as operações. Cada nó representa uma função ou operação específica \cite{dataflow-procedure, 48862}.

A programação de fluxo de dados permite, por natureza, um alto grau de paralelismo. Diferentemente da arquitetura tradicional de von Neumann, onde as instruções são executadas em ordem e o controle de fluxo é centralizado, na PFD a execução ocorre de maneira assíncrona e descentralizada. Cada \textit{node} (nó) é executado assim que todos os dados necessários para sua operação estejam disponíveis. Esse modelo elimina a necessidade de um controle centralizado de fluxo, permitindo que múltiplas operações sejam realizadas simultaneamente, com melhor aproveitamento dos recucos computacionais disponíveis \cite{48862}.

O alto paralelismo natural da PFD faz com que, na área de edição de imagem, o fluxo de edição se torne descentralizado e não linear. Em programas tradicionais de edição de imagem, o usuário segue uma ordem linear de operações, fazendo com que cada mudança dependa da anterior. Com a implementação de um modelo Dataflow em um programa de edição de imagem, esse foco muda e se torna descentralizado, permitindo que o usuário edite uma imagem de diversas maneiras, sem depender de uma sequência rígida.

Programas desenvolvidos sob este paradigma são estruturados por meio de grupos de nodes (ou blocos), interligados por conexões que representam o fluxo de dados. Cada node possui portas de entrada e/ou saída, que pode exercer diferentes funções dentro do sistema, demonstrado na figura ~\ref{fig:dataflow} De forma geral, esses nós são classificados em três categorias principais, sendo elas: blocos de entrada, blocos de processamento e blocos de saída. As informações percorrem o programa seguindo essas conexões, introduzidos pelos blocos de entrada, sendo transformados conforme passam pelos blocos processadores, até atingirem os destinos finais nos blocos de saída, que coletam os resultados \cite{dataflow}.

\begin{figure}[ht]
    \centering
    \includegraphics[width=1\textwidth]{imagens/dfpex.png}
    \caption{
    Um simples programa (a) e o seu equivalente em no formato Dataflow (b)}
    \label{fig:dataflow}
    \cite{dataflow}
\end{figure}

\section{Justificativa}

Este trabalho propõe a criação de um aplicativo de edição de imagens baseado em grafos para a modelagem do fluxo da edição, tornando possível não apenas compreender e documentar o processo, mas também oferecer ao usuário final, que não possui o conhecimento técnico necessário para operar um aplicativo de edição mais avançado, meios mais intuitivos de modificar e experimentar sequências de transformações. Dessa forma, obtém-se uma solução de edição descentralizada, em que a criatividade é o único limite para a edição. Além disso, a adoção dessa metodologia pode servir como base para o desenvolvimento de aplicações que automatizem ou recomendem etapas de edição.

\section{Motivação}

A criação deste aplicativo origina-se principalmente na dificuldade de utilização de softwares profissionais, que, apesar de sua popularização, ainda apresentam uma barreira de entrada significativa para usuários não técnicos. Esses softwares tendem a não ser amigáveis para novos interessados na área, exigindo muitas vezes um verdadeiro “curso” para que se aprenda a utilizá-los, se tornando mais um requerimento do que apenas um complemento para usuários já familiarizados que querem elevar o nível de conhecimento.

\section{Objetivos}

Na presente seção serão expostos os objetivos do trabalho, tanto gerais quanto específicos.

\subsection{Objetivos Gerais}

O objetivo principal deste projeto é a criação de um aplicativo de edição de foto baseado em nodes, com a sua principal finalidade sendo a implementação do diagrama de fluxo de dados, de uma forma visual, de forma que o usuário possa editar fotos e/ou imagens de uma forma disruptiva e não linear.

Diferindo da tradicional forma de edição fotográfica sendo, uma forma linear de editar as imagens, com instruções uma após a outra, assim limitando o usuário a seguir um padrão ao invés de dar a liberdade para que o usuário possa, da forma que preferir, editar a imagem.

\subsection{Objetivos Específicos}

\begin{itemize}
    \item Implementar uma UI interativa e de fácil compreensão, que facilite a manipulação e configuração de componentes, sem a necessidade de grande um conhecimento técnico.
    \item Implementar o paradigma de programação de fluxo de dados, permitindo que usuários construam pipelines de processamento de forma visual e modular.
    \item Implementar criação de Shaders para edições de imagens e/ou criação de imagens dinâmicas.
    \item Implementar \textit{nodes} funcionais capazes de realizar mudanças nas imagens, atendendo os parâmetros e requisitos definidos pelo usuário, independentemente da posição ou ordem no grafo.
    \item Implementar \textit{nodes} capazes de criar Shaders, tanto em versões de código, feitos pelo usuário, quanto nodes de código já previamente feitos pelo programa.
    \item Implementar um servidor websocket para sessões colaborativas, em que dois ou mais usuários possam editar a mesma imagem ao mesmo tempo.
\end{itemize}

\section{Organização dos Capítulos}

Na presente seção será apresentada a organização deste trabalho, divido em seções, iniciando pela introdução, seguida pela fundamentação teórica, trabalhos relacionados, desenvolvimento e conclusão do trabalho.

%Capítulo 0X - Fundamentação Teórica
\chapter[Fundamentação]{Fundamentação Teórica}

Na presente seção serão abordados tópicos específicos sobre programação e processamento de imagens que são fundamentais para a compreensão do presente trabalho. Inicialmente, será apresentado, de forma breve, os principais elementos que compõem a base do sistema desenvolvido, incluindo a linguagem escolhida para o desenvolvimento do aplicativo, bibliotecas, aspectos de UI/UX e sua importância para este projeto, bem como o paradigma de desenvolvimento adotado, sendo considerado um dos diferenciais do projeto.

Posteriormente, será abrangido o tema de processamento de imagens, em maior profundidade,e suas aplicações neste trabalho, com ênfase nos conceitos práticos que embasam sua aplicação no projeto. Por fim, será discutido o processo de criação e utilização dos \textit{Shaders}, explicando seu funcionamento e importância.

\section{Python}

Python, uma linguagem de programação alto nível e propósito geral, amplamente reconhecida por sua simplicidade sintática, legibilidade e ampla comunidade de usuários. Sua praticidade e versatilidade tornam-no uma escolha popular em diversas áreas da computação, incluindo, mas não se limitando a, ciência de dados, automação desenvolvimento web e processamento de imagens \cite{Python}.

Nesse contexto, o Python se destaca por oferecer um ecossistema robusto de bibliotecas especializadas, como OpenCV, NumPy, Scikit-image, Pillow, entre outras, que permitem desde simples operações de manipulação de imagem até o desenvolvimento de algorítimos avançados de análise, restauração e aperfeiçoamento visual.
Essa ampla gama de recursos torna a linguagem uma ferramenta altamente eficaz para projetos que exigem processamento e tratamento de imagens digitais, como é o caso deste trabalho \cite{PillowDocs, PyOpenCV, NumPy, scikit-image}.

\section{Bibliotecas}

Bibliotecas, no sentido tradicional, são locais onde inúmeros livros são  armazenados para consulta, permitindo que os leitores adquiram conhecimento em diversas áreas e sobre os mais variados temas. As bibliotecas de programação seguem um conceito semelhante sendo elas, coleções de funções, rotinas ou módulos reutilizáveis, escritas em determinada linguagem de programação \cite{GeeksForGeeks}. E assim como livros, aonde releituras ou adaptações são constantemente criadas, as bibliotecas também passam por esse mesmo processo, dependendo da licença de uma biblioteca, ela pode ser reescrita em outra linguagem, ou adaptada como um binding ou wrapper, ou uma nova versão pode ser refeita, por uma pessoa diferente ou pelo criador original da mesma \cite{bind-wrap}.

Nesta seção serão abordadas as bibliotecas utilizadas neste projeto, tanto de processamento de imagens como de interface de usuário.

\subsection{Dear PyGui}

O Dear PyGui (DPG) é uma biblioteca para Python rápida e poderosa para a criação de interfaces gráficas, originalmente desenvolvida em C++ sob o nome de Dear ImGui (\textit{Immediate-Mode Graphical User Interface}). Devido a crescente popularidade da linguagem Python, a biblioteca foi adaptada para esse ambiente, mantendo os princípios de interface em modo imediato e oferecendo uma API simplificada e eficiente \cite{DearPyGui}.

O Dear PyGui (DPG) segue o paradigma de GUI em modo imediato (“Immediate mode GUI”) que, ao invés de manter uma árvore de componentes entre interações, como ocorre no modo retido (retained mode), o framework desenha a interface a cada frame com base no estado atual da aplicação \cite{DPGdocs, ImGui, immediate-modeGUI}.

No modo retido (\textit{retained mode}), frameworks que utilizam esse estilo armazenam os elementos da interface, como botões, janelas e caixas de texto, em uma árvore ou hierarquia de componentes. A biblioteca gerencia automaticamente os estados dos elementos e atualiza apenas aqueles que sofrem alterações, ou seja, elementos cujo estado, cor, tamanho ou outras propriedades foram modificados pelo usuário ou por outro componente.

Um exemplo de tecnologia amplamente conhecida por utilizar esse método é o CSS (\textit{Cascading Style Sheets}), que mantém o estado de cada elemento em uma estrutura hierárquica e realiza a atualização somente do elemento que foi interagido ou afetado por alterações \cite{MDNdocs}.

Entretanto, um dos problemas associados a esse modelo de interface é justamente o gerenciamento de estados. Em aplicações com muitos componentes, esse controle pode se tornar difícil de manter, gerando error que passam despercebidos. Por exemplo, ao deletar um componente e, inadvertidamente, tentar modificar o estado do mesmo, cuja referência ainda permaneça em algum ponto da hierarquia, isso acabaria causando em um erro, pois, o elemento modificado não estará mais presente na hierarquia.

No modo imediato, característico da biblioteca Dearpygui, exige que os elementos sejam desenhados a cada ciclo de atualização, ou a cada frame. Isso oferece alto controle e facilita a criação de interfaces altamente dinâmicas e responsivas, especialmente em tempo real \cite{retain-immediate}. Computadores mais fracos, em relação a CPU, acabam sofrendo com aplicações que utilizam esse modo de interface gráfica.

\subsection{PILLOW}

A biblioteca Pillow é uma continuação aprimorada e mantida até os dias atuais, originalmente desenvolvida sob o nome de PIL (Python Imaging Library) feita por Fredrik Lundh \cite{Pillow-Docs}, amplamente utilizada para processamento de imagens em Python. O pillow surgiu como uma fork do PIL, garantindo uma compatibilidade retroativa e adicionando suporte a recursos modernos, além da manutenção continua da comunidade \textit{open source}.

A biblioteca oferece uma ampla variedade de funcionalidades, como a abertura e a possibilidade de salvar imagens em diversos formatos como por exemplo, JPEG, PNG, BMP GIF e TIFF, além de suas outras funcionalidades como redimensionamento de imagens, corte, filtros alteração de cores entre outras \cite{PillowDocs}.

\subsection{PyOpenGL}

O PyOpenGL é um conjunto de bindings (ligações) em Python para a biblioteca gráfica OpenGL, permitindo que programas escritos em Python façam o uso direto dessa API de gráficos 2D e 3D. Por meio dela, é possível acessar praticamente todas as funcionalidades da API OpenGL, inclusive extensões modernas como shaders em GLSL. O PyOpenGL é multiplataforma, e é amplamente utilizado para o desenvolvimento de visualizações científicas, simulações e jogos \cite{PyOpenGl}.

\section{Shaders}

Shaders são pequenos programas que são executados diretamente na GPU (Graphical Process Unit), responsáveis por controlar o processamento de vértices e pixels durante o pipeline gráfico. Escritos geralmente em linguagens como GLSL (OpenGL Shading Language), permitem criar efeitos visuais complexos, como iluminação dinâmica, mapeamento de texturas, sombras e reflexos. Os tipos mais comuns de shaders são dois, \textit{vertex shaders}, que manipulam posições e propriedades dos vértices, e os \textit{fragment shaders} que determina a cor final de cada fragmento (pixel) renderizado \cite{GLSLang}.

\section{UI/UX}

A área de \textit{User Interface/User Experience} representa um aspecto essencial no desenvolvimento de qualquer software sendo, site ou aplicativo. Uma interface mal projetada pode comprometer significativamente a usabilidade da aplicação, deixando-a confusa e pouco intuitiva.

No contexto deste trabalho, a atenção à UI/UX foi uma das prioridades no processo de desenvolvimento. A proposta do aplicativo de ser de fácil utilização e rápido aprendizado, busca se distanciar de soluções complexas, constantemente acompanhadas com funcionalidades ocultas ou mal estruturadas.

\section{Dataflow}

"Dataflow programming (DFP) é um paradigma de programação, onde a execução do programa é conceitualizada como dados fluindo por uma série de operações ou transformações" \cite{devopedia}. Diferente dos paradigmas tradicionais imperativos, onde o controle do fluxo é determinando pela sequência explícita de instruções, na PFD o foco está no fluxo dos dados e na maneira como eles propagam pelos nós (ou blocos funcionais) do sistema. Nesse modelo, cada operação ou nó executa seu processamento assim que todos os dados necessários para sua execução estão disponíveis, independentemente de uma ordem de execução global. Isso torna a PFD naturalmente concorrente e paralelizável, uma vez que operações independentes podem ser executadas simultaneamente sempre que seus dados estiverem disponíveis \cite{lee-parks}.

Entre as principais características da PFD, destaca-se a execução orientada a dados. A execução de cada nó é acionada pela chegada de dados completos, dispensando o uso de estruturas tradicionais de controle de fluxo como por exemplo loops ou comandos condicionais. Ademais, como os nós são independentes, podem ser mapeados facilmente para diferentes threads ou até mesmo para processadores distintos, explorando ao máximo arquiteturas multicore e distribuídas \cite{ackerman.w.b}. O modelo PFD contribui para a tolerância a latências, pois em sistemas distribuídos ou heterogêneos diferentes nós podem prosseguir com o processando dados independentemente do tempo de processamento de outros nós. \cite{malteschwarzkopf}

O Dataflow possui múltiplas implementações. Uma das implementações clássicas e mais simples é o modelo baseado em tokens, no qual os dados fluem pelos grafos na forma de tokens que transitam por filas do tipo FIFO (first in, first out) \cite{JohnstonHannaMillar, malteschwarzkopf}. Quando os dados necessários estão disponíveis nas entradas de um nó, ele é denominado de um nó \textit{fireable}, ou seja, o node está pronto para ser executado. Um nó \textit{fireable} será executado após um determinado intervalo de tempo desde o recebimento de seus dados. Passando esse período, ele é ativado, seus tokens de saída avançam para o próximo nó, o qual poderá então se tornar \textit{fireable}. O nó anterior, por sua vez, perde a denominação de \textit{fireable}, retorna ao seu estado padrão e aguarda até que o processo possa se repetir.

De forma geral, programas de fluxo de dados são representados por grafos direcionados, nos quais os nós (ou vértices) representam operações ou transformações sobre os dados, enquanto arestas representam canais por onde os dados circulam entre operações. Esse grafo pode ser estático ou dinâmico. Em grafos estáticos, a estrutura do fluxo é conhecida em tempo de compilação, o que facilita otimização e analises. Entretanto, certos programas não podem ser representados por esse tipo de grafo, pois laços (loops), por exemplo, só podem ser modelados caso o número de iterações já seja conhecido em tempo de compilação. Já os grafos dinâmicos, oferecem uma maior flexibilidade, permitindo que nós com múltiplos arcos sejam diferenciados por cores, em que cada cor é ativada exclusivamente para seu arco específico \cite{JohnstonHannaMillar}.

Em comparação com a tradicional programação imperativa, a PFD não depende de uma sequencia explícita de instruções, mas sim das dependências entre dados. Já em relação à programação funcional, embora ambos incentivem programas sem efeitos colaterais e baseados em composição, a PFD enfatiza o movimento e a dependência de dados entre operações, enquanto a programação funcional foca principalmente em funções puras e transformações sobre coleções \cite{lee-parks}.

A adoção desse paradigma traz diversas vantagens. O paralelismo surge de forma implícita, eliminando a necessidade de escrever explicitamente código de sincronização, pois o próprio modelo organiza as dependências de dados. Além disso, muitas implementação de PFD garantem determinismo, assegurando que, dado o mesmo conjunto de entradas, o programa sempre produzirá o mesmo resultado, independentemente da ordem em que as operações internas são realizadas.

\section{Teoria dos grafos}

A teoria dos grafos é um ramo da matemática discreta que estuda as relações entre objetos representados por meio de estruturas chamadas grafos. Um grafo é definido, formalmente, como um par ordenado \(G = (V, A)\), onde \(V\) é um conjunto de vértices (ou nós), e \(A\) é o conjunto de arestas que conectam pares de vértices. Dependendo do problema ou da aplicação, grafos podem ser direcionados ou não direcionados \cite{graph-theory-1}.

Nos grafos não direcionados, as conexões entre vértices (ou nós) não possuem direção ou sentido específico; já nos grafos direcionados, cada aresta possui sentido definido, sendo por isso também chamado de arco.

A origem formal do estudo dos grafos é atribuída ao matemático Leonhard Euler, que em 1736 discutiu o famoso enigma das pontes de Königsberg, estabelecendo assim os fundamentos do que hoje é conhecido como teoria dos grafos. Desde então, o campo expandiu-se substancialmente, encontrando aplicações em diversas áreas, como ciência da computação, engenharia elétrica, biologia, logística e redes sociais \cite{graph-theory-2}.

%Capítulo 0X - Trabalhos Relacionados
\chapter[Trabalhos Relacionados]{Trabalhos Relacionados}

Na presente seção serão apresentados softwares de edição de imagem baseados em nodes (ou grafos). Serão analisados os principais aspectos desses programas, destacando como cada um implementa a modelagem de fluxo de edição, quais recursos são oferecidos aos usuários e a praticidade de uso. O objetivo é identificar pontos fortes e limitações, de modo a fornecer uma base comparativa para a proposta desenvolvida neste trabalho.

\section{Gimel Studio}

O Gimel Studio é um software de código aberto destinado à edição de imagens baseado em nodes. Desenvolvido na linguagem de programação Dart e utilizando o framework Flutter para a construção da interface gráfica, o Gimel Studio apresenta similaridades visuais com editores tradicionais, como o Adobe Photoshop. A Figura~[\ref{fig:gimelstudio}] apresenta a interface do Gimel Studio, que é caracterizada por sua semelhança com editores convencionais, facilitando a adaptação de usuários familiarizados com esse estilo de interface.

Atualmente o Gimel Studio se encontra em estágio inicial de desenvolvimento. O repositório oficial fornece instruções detalhadas para a execução do projeto localmente. Até o presente momento, existe somente um lançamento de uma versão alpha, feita em Janeiro de 2022. Desde então, o desenvolvimento não apresenta atualizações significativas.

O Gimel Studio oferece um conjunto básico de nodes/módulos de edição, incluindo ajustes de brilho, contraste, saturação, desfoque e conversão para escala de cinza. A interface do usuário é intuitiva, permitindo a criação e conexão de nodes de forma visual. No entanto, o software não possui funcionalidades avançadas e uma comunidade pouco ativa, o que limita seu potencial de crescimento e adoção.

Em comparação com o aplicativo proposto neste trabalho, o Gimel Studio apresenta uma interface mais tradicional, semelhante a editores de imagem convencionais, o que pode ser vantajoso para usuários acostumados a esse estilo. No entanto, o conjunto limitado de funcionalidades e a falta de suporte à colaboração em tempo real representam desvantagens significativas. O aplicativo desenvolvido neste trabalho busca oferecer uma experiência mais robusta e colaborativa, com uma gama maior de módulos de edição e uma interface mais refinada, visando atender a um público mais amplo, incluindo usuários sem conhecimento técnico avançado.

\begin{figure}[H]
    \centering
    \includegraphics[width=1\textwidth]{imagens/gimelstudio.png}
    \caption{Imagem do Gimel Studio}
    \label{fig:gimelstudio}
    \cite{gimelstudio}
\end{figure}


\section{Cresliant}

O Cresliant é um software de código aberto destinado à edição de imagens baseado em nodes. Desenvolvido na linguagem de programação Python, o Cresliant destaca-se por oferecer uma interface gráfica intuitiva e acessível ao usuário final, frequentemente referida como \textit{User-Friendly}. Entre as tecnologias utilizadas em sua criação, destaca-se o uso do \textit{Poetry}, para o gerenciamento eficiente de dependências e ambientes virtuais, bem como a biblioteca PILLOW (fork da Python Imaging Library - PIL), que desempenha papel crucial no processamento e manipulação de imagens do software. A Figura~[\ref{fig:cresliant}] apresenta a interface do Cresliant, que é caracterizada por sua simplicidade e facilidade de uso.

Apesar de o repositório do Cresliant, no Github, não apresentar atualizações recentes, sendo o último \textit{commit} registrado há aproximadamente dois anos, e a versão mais recente datada de dezembro de 2023, o software permanece funcional e estável, cumprindo de maneira eficaz o propósito a que se destina.

O Cresliant oferece, por padrão, poucos nodes/módulos de edição. No entanto, possibilita que o usuário crie novos nodes/módulos e os adicione como submódulos/\textit{addons} externos, sem a necessidade de modificações diretas no código-fonte. Essa abordagem promove uma certa liberdade ao usuário final, mas ao mesmo tempo limita a funcionalidade apenas àqueles que possuem conhecimento 
técnico suficiente para realizar tal integração de novas funcionalidades.

O aplicativo Cresliant foca em fornecer uma experiência de edição de imagens baseada em nodes, priorizando a simplicidade e a facilidade de uso. Em comparação com o aplicativo proposto neste trabalho, o Cresliant apresenta uma interface mais básica e um conjunto limitado de funcionalidades nativas. O aplicativo desenvolvido neste trabalho busca oferecer uma experiência mais robusta, com uma gama maior de módulos de edição e uma interface mais refinada, visando atender a um público mais amplo, incluindo usuários sem conhecimento técnico avançado, além de oferecer uma funcionalidade única, sendo a colaboração em tempo real entre múltiplos usuários, o que não está presente no Cresliant.

\begin{figure}[H]
    \centering
    \includegraphics[width=1\textwidth]{imagens/cresliant.png}
    \caption{Imagem do Cresliant}
    \label{fig:cresliant}
    \cite{cresliant}
\end{figure}


\section{PixiEditor}

O PixiEditor é software de código aberto destinado à criação de gráficos procedurais por meio de nodes. Desenvolvido na linguagem de programação Csharp, o PixiEditor destaca-se por sua robustez e ampla gama de funcionalidades. Além do editor de gráficos procedurais, o software oferece recursos adicionais, como criação de animações 2D, edição tradicional de imagens 2D, entre outras ferramentas, sendo ainda compatível com múltiplas plataformas. A Figura~[\ref{fig:pixieditor}] apresenta a interface do PixiEditor, que é caracterizada por sua complexidade e riqueza de recursos.

Atualmente o PixiEditor encontra-se com o desenvolvimento ativo, com sua ultima versão lançada no dia 10 de setembro de 2025. Com sua vasta gama de funcionalidades, o PixiEditor se apresenta como uma alternativa viável para profissionais e entusiastas que buscam uma solução completa para criação e edição gráfica.

Em comparação com o aplicativo proposto neste trabalho, o PixiEditor oferece uma gama mais ampla de funcionalidades, indo além da edição de imagens baseada em nodes. A interface do PixiEditor é mais complexa, refletindo a diversidade de recursos oferecidos, o que pode representar um desafio para usuários iniciantes. Em contrapartida, o aplicativo proposto neste trabalho prioriza a simplicidade e a facilidade de uso, visando atender especialmente usuários sem conhecimento técnico avançado, e se concentra exclusivamente na edição de imagens por meio de nodes, proporcionando uma experiência mais direta e acessível.

\begin{figure}[H]
    \centering
    \includegraphics[width=1\textwidth]{imagens/pixi.png}
    \caption{Imagem do Pixi Editor}
    \label{fig:pixieditor}
    \cite{pixieditor}
\end{figure}


\section{Photoshop}

O Adobe Photoshop é um dos mais poderosos e populares editores de imagens e fotos do mundo, com diversas ferramentas para edição, criação e manipulação de imagens. 
Desenvolvido pela Adobe, é amplamente utilizado por profissionais de design gráfico, fotografia, ilustração digital e outras áreas relacionadas. A Figura~[\ref{fig:photoshop}] apresenta a interface do Photoshop, que é caracterizada por sua complexidade e riqueza de recursos.

O Photoshop oferece uma ampla gama de recursos, incluindo ferramentas de seleção, camadas, máscaras, filtros, ajustes de cor, suporte a plugins, automação por scripts, 
integração com outros softwares da Adobe (como Illustrator e After Effects), além de suporte a múltiplos formatos de arquivo.

O Photoshop não é baseado em nodes, mas sim em uma interface tradicional de edição de imagens. 
Apesar disso, é um software amplamente reconhecido e utilizado na indústria de design gráfico e edição de imagens, sendo referência em qualidade, flexibilidade e produtividade.

O software é proprietário, com licença comercial, disponível para Windows e macOS. Possui uma comunidade global, extensa documentação, suporte técnico oficial e uma vasta gama de plugins e integrações. 
O fluxo de trabalho é altamente profissional, com recursos avançados de automação e personalização, indicado para usuários avançados e profissionais.

Em contraste com o aplicativo proposto neste trabalho, o Photoshop oferece uma gama muito mais ampla de funcionalidades e recursos avançados, atendendo a um público profissional. O aplicativo desenvolvido neste trabalho foca na simplicidade e acessibilidade, visando atender usuários sem conhecimento técnico avançado, com ênfase na edição de imagens por meio de nodes e colaboração em tempo real, características que não estão presentes no Photoshop.

\begin{figure}[H]
    \centering
    \includegraphics[width=1\textwidth]{imagens/photoshop.jpg}
    \caption{Imagem do Photoshop}
    \label{fig:photoshop}
    \cite{photoshop}
\end{figure}


\section{MS Paint}

O MS Paint é um editor e criador de imagens e desenhos tradicional, integrado ao sistema operacional Windows desde suas primeiras versões. É um software simples, gratuito e voltado para tarefas básicas de edição e criação de imagens. A Figura~[\ref{fig:mspaint}] apresenta a interface do MS Paint, que é caracterizada por sua simplicidade e facilidade de uso.

O MS Paint oferece ferramentas elementares como pincel, lápis, balde de tinta, formas geométricas, seleção e recorte, sendo indicado para usuários iniciantes ou para edições rápidas e descomplicadas.

O MS Paint não possui suporte a nodes, plugins ou extensões, e sua interface é extremamente simples, sem recursos avançados de edição. Não há integração com outros softwares, mas sua facilidade de uso e disponibilidade o tornam uma ferramenta acessível para o público geral, sendo frequentemente utilizado para tarefas cotidianas, anotações rápidas e ilustrações simples.

O MS Paint é uma simples ferramenta de edição de imagens, focada em tarefas básicas e acessibilidade. Em contraste, o aplicativo proposto neste trabalho oferece uma abordagem mais avançada e flexível para edição de imagens, utilizando nodes e permitindo colaboração em tempo real, características que não estão presentes no MS Paint. O aplicativo desenvolvido visa atender a um público mais amplo, incluindo usuários sem conhecimento técnico avançado, mas com necessidades de edição mais complexas do que aquelas atendidas pelo MS Paint.

\begin{figure}[H]
    \centering
    \includegraphics[width=1\textwidth]{imagens/paint.jpg}
    \caption{Imagem do MS Paint \\ (Fonte: autoria própria)}
    \label{fig:mspaint}
\end{figure}


\section{Comparativo}

A Tabela [\ref{tab:comparativo}] apresenta um comparativo entre os softwares analisados nesta seção, destacando suas principais características, funcionalidades e limitações.

\begin{table}[H]
    \centering
    \begin{tabular}{|l|c|c|c|c|}
        \hline
        \textbf{Software} & \textbf{Baseado em Nodes} & \textbf{Funcionalidades} & \textbf{Web} & \textbf{Colaboração} \\
        \hline
        Gimel
        Studio & Sim & Básico & Não & Não \\
        \hline
        Cresliant & Sim & Básico & Não & Não \\
        \hline
        Pixi
        Editor & Sim & Avançado & Não & Não \\
        \hline
        Adobe
        Photoshop & Não & Muito Avançado & Não & Não \\
        \hline
        MS
        Paint & Não & Muito Básico & Não & Não\\
        \hline
        Aplicativo
        Proposto & Sim & Básico & Sim & Sim \\
        \hline
    \end{tabular}
    \caption{Comparativo entre softwares de edição de imagem}
    \label{tab:comparativo}
\end{table}

%Capítulo 0X - Metodologia
\chapter[Implementação do PhotoEditor]{Implementação do PhotoEditor}
\renewcommand{\thelstlisting}{\arabic{lstlisting}}

Na presente seção, será descrita a metodologia utilizada para a realização deste trabalho, separada por tópicos que englobam todo o processo de criação do projeto.

O processo de desenvolvimento foi dividido em etapas bem definidas, abrangendo o levantamento de requisitos, a arquitetura do sistema, implementação dos módulos, os resultados e as limitações.


\section{Visão Geral do Projeto}

Esta seção apresenta uma visão geral do projeto, destacando seus objetivos principais, o público-alvo e o contexto em que será aplicado. O projeto visa desenvolver uma solução eficiente e escalável para atender às necessidades identificadas durante o levantamento de requisitos.

A criação de um software de edição de imagens baseado em nodes tem como objetivo proporcionar uma ferramenta intuitiva e poderosa para artistas digitais, designers gráficos e entusiastas da edição de imagens. O público-alvo inclui tanto profissionais experientes quanto iniciantes que buscam uma alternativa flexível aos editores tradicionais.

\section{Requisitos}

Nesta seção, serão detalhados os requisitos do sistema, que guiaram o desenvolvimento do projeto. Os requisitos foram coletados por meio de análise de softwares similares.

\begin{table}[H]
    \centering
    \begin{tabular}{|c|p{12cm}|}
        \hline
        \textbf{Requisito} & \textbf{Descrição} \\
        \hline
        R1 & Interface gráfica baseada em nodes para edição de imagens. \\
        \hline
        R2 & Suporte a múltiplos formatos de imagem (JPEG, PNG, BMP, etc.). \\
        \hline
        R3 & Funcionalidades básicas de edição (redimensionamento, ajuste de cores, filtros). \\
        \hline
        R4 & Capacidade de salvar e carregar projetos. \\
        \hline
        R5 & Extensibilidade para adicionar novos módulos de edição. \\
        \hline
        R6 & Desempenho otimizado para manipulação de imagens grandes. \\
        \hline
        R7 & Compatibilidade com múltiplas plataformas (Windows, Linux). \\
        \hline
        R8 & Servidor websocket para sessões colaborativas em tempo real. \\
        \hline
    \end{tabular}
    \caption{Requisitos do Sistema}
    \label{tab:requisitos}
\end{table}

\section{Arquitetura do Sistema}

A arquitetura central do sistema implementa um paradigma de programação visual onde operações de processamento são representadas como nós conectáveis em um grafo direcionado. Cada nó encapsula uma operação específica de processamento de imagem, possuindo entradas (inputs) que recebem dados de outros nós, parâmetros configuráveis que controlam o comportamento da operação e saídas (outputs) que produzem dados para o consumo de nós subsequentes.

Esta abordagem oferece vantagens significativas sobre interfaces tradicionais: permite visualização clara do fluxo de dados, facilita a experimentação através de reconexão rápida de componentes e proporciona reutilização de subgrafos em diferentes contextos. A natureza visual do paradigma torna o processo de edição mais intuitivo, especialmente para usuários que pensam em termos de pipelines de processamento.

A arquitetura de comunicação entre componentes é baseada em eventos, onde mudanças em um nó disparam atualizações nos nós subsequentes. Esta abordagem reativa garante que o estado do sistema permaneça consistente, com atualizações propagadas automaticamente através do grafo conforme o usuário interage com a interface.

A Figura [\ref{fig:diagrama-arquitetura}] ilustra a arquitetura geral do sistema, destacando os principais componentes e suas interações, o usuário interage com a interface gráfica, que se comunica com o motor de processamento de imagens, o processamento é realizado por módulos especializados, dependendo do tipo do nó, após o processamento da imagem, o mesmo nó também envia um evento ao servidor websocket, caso o usuário esteja em uma sessão colaborativa, para que as outras instâncias do programa sejam atualizadas de acordo com as mudanças feitas pelo usuário, após a imagem ser processada, a interface gráfica é atualizada com a mudança realizada. O sistema de colaboração em tempo real, também é responsável por receber informações vindas do servidor para atualizar a interface gráfica do usuário que não realizou a modificação em sua sessão local.

\begin{figure}[H]
    \centering
    \includegraphics[width=1\textwidth]{imagens/diagrama_arquitetura.png}
    \caption{Diagrama da arquitetura do sistema}
    \label{fig:diagrama-arquitetura}
\end{figure}

\subsection{Interface Gráfica}

O desenvolvimento da interface gráfica foi realizado utilizando a biblioteca Dear PyGui (DPG), um \textit{framework} moderno de interface gráfica que oferece renderização acelerada por hardware e suporte nativo a editores baseados em nós. Esta escolha diferencia-se de \textit{frameworks} tradicionais como TKinter ou PyQt por oferecer performance superior em aplicações gráficas intensivas e APIs simplificadas para criação de editores visuais.

\begin{figure}[H]
    \centering
    \includegraphics[width=1\textwidth]{imagens/canva_aplicativo.png}
    \caption{Imagem da interface do programa}
    \label{fig:interface-nodes}
\end{figure}

A integração de outras bibliotecas com o Dear PyGui requer uma compreensão profunda do seu modelo de programação, modo imediato, pois o DPG opera em um loop de renderização contínuo onde a interface é atualizada frame a frame, diferindo de \textit{frameworks} baseados em eventos tradicionais como Tkinter ou PyQt.

A biblioteca abstrai muitos detalhes de baixo nível, permitindo que o desenvolvedor foque na lógica da aplicação ao invés de detalhes de renderização. Isso acelera o desenvolvimento e reduz a complexidade do código.

A criação de nós na interface gráfica é realizada utilizando os \textit{context managers} do Python, permitindo assim a criação de objetos de forma eficiente. Os \textit{context managers} são utilizados para garantir que os elementos criados sejam devidamente destruídos após o uso, garantindo a integridade do sistema e prevenindo que erros que acontecem durante a execução do programa não sejam ignorados ou que façam o programa fechar de forma abrupta.

\subsection{Design da Interface}

A interface principal do aplicativo utiliza uma janela única maximizada que aproveita todo o espaço disponível da tela. Esta abordagem elimina distrações e fornece o máximo de espaço para o canvas de edição.

O aplicativo implementa menus de funcionalidades simples e diretos, organizados logicamente: menu \textit{File} para operações de projeto, menu \textit{Help} para documentação e links externos, e menu \textit{Dev} para ferramentas de desenvolvimento e debug.

A escolha por uma abordagem minimalista na interface visa maximizar a área de trabalho disponível para o usuário, reduzindo elementos visuais desnecessários que possam distrair ou ocupar espaço valioso na tela, evitando que o usuário perca tempo descobrindo a funcionalidade de cada botão disponível na tela e consiga focar ao máximo na edição da imagem.

Para o \textit{design} da interface foi utilizado o site \textit{figma.com}, uma ferramenta de prototipagem e design colaborativo baseada na web. O Figma permite a criação rápida de protótipos interativos, facilitando a visualização e iteração do design da interface antes da implementação.

\subsection{Canvas}

O canvas do aplicativo implementa uma área de trabalho infinita, permitindo que o usuário posicione os nós de forma livre sem restrições. A adoção de tons neutros e suaves reduz a fadiga ocular, proporcionando um ambiente confortável para o trabalho prolongado.

A escolha da paleta de cores leva em conta também a acessibilidade, utilizando contrastes adequados para garantir que elementos possam se distinguir claramente perante outros elementos, facilitando a identificação rápida de cada nó de edição presente na tela.

\subsection{Processamento de Imagens}

A biblioteca Pillow (PIL) foi selecionada como base para operações de processamento de imagem devido à sua estabilidade, performance otimizada e suporte abrangente a formatos de arquivo. Pillow oferece operações fundamentais como conversão de espaços de cor, redimensionamento, filtros básicos e manipulação de canais, formando a base sobre a qual algoritmos mais complexos são construídos.

A integração entre Pillow e Dear PyGui requer conversão cuidadosa entre formatos de dados, onde imagens são convertidas de objetos PIL para arrays numpy e subsequentemente para texturas GPU para renderização eficiente na interface gráfica.

Além do Pillow, foi utilizada a biblioteca ModernGL para operações mais complexas de processamento de imagens como o uso de \textit{fragment shaders} e \textit{vertex shaders}, que permitem a manipulação direta de pixels na GPU, proporcionando performance superior para efeitos visuais avançados.

Embora a implementação atual seja principalmente \textit{single-threaded} para simplicidade, a arquitetura suporta extensão para processamento paralelo. Nós independentes em um grafo podem ser processados simultaneamente, e operações computacionalmente intensivas podem utilizar \textit{threading} para manter a responsividade da interface. Com a utilização de um sistema de distribuição de tarefas, o sistema pode ser escalado para aproveitar múltiplos núcleos de CPU.

\subsection{Tipos de Nós}

Na presente seção será detalhada a implementação dos principais tipos de nós desenvolvidos para o sistema, suas funcionalidades e características.

\subsubsection{Nó Base}

A arquitetura de nós é fundamentada em uma classe base abstrata, chamada de \textit{NodeCore}, que define a interface comum para todos os tipos de nós. Esta classe implementa funcionalidades essenciais como gerenciamento de estado, serialização, conexões entre nós, integração com o sistema de interface gráfica, posicionamento automático e atualizações automáticas no servidor websocket.

A implementação da classe base estabelece padrões nos quais nós subsequentes devem seguir, garantindo consistência e reusabilidade. O método \textit{initialize()} é responsável pela criação visual do nó na interface, enquanto o método \textit{run()} é responsável pela lógica de processamento específica de cada nó, como demonstrado no código [\ref{lst:nodecore}]. A classe base também implementa um método para gerenciar a criação de nós no servidor websocket, garantindo a singularidade de nós por meio de um sistema de tags.

\begin{lstlisting}[language=Python, caption={Exemplo de implementação da classe base NodeCore}, label={lst:nodecore}]
class NodeCore:
    def __init__(self):
        self.settings = {}
        self.last_node_id = None
        self.counter = 0

    def initialize(self, parent=None, node_tag: str | None = None, pos: list[int] | None = None):
        # Implementacao de criacao visual do no
        pass
    
    def run(self, image: Image.Image, tag: str) -> Image.Image:
        # Metodo abstrato para o processamento
        raise NotImplementedError
\end{lstlisting}

A classe base utiliza um sistema de tags, permitindo a identificação única de cada nó, garantindo a existência de múltiplas instâncias do mesmo tipo de nó sem conflitos. A classe base também contém um método de posicionamento automático para a prevenção de sobreposição de nós na interface.

\subsubsection{Monocromo}

A implementação do nó de conversão monocromática exemplifica a arquitetura de processamento adotada. Este nó converte imagens coloridas para escala de cinza utilizando o método padrão de conversão do Pillow, que preserva informações de luminância perceptual.

\begin{lstlisting}[language=Python, caption={Implementação do método run do nó monocromático}, label={lst:monocromo}]
def run(self, image: Image.Image, tag: str) -> Image.Image:
    return image.convert("L").convert("RGBA")
\end{lstlisting}

Ao analisarmos a implementação, a conversão passa por duas etapas: primeiro converte para escala de cinza ("L"), depois retorna ao formato RGBA para manter compatibilidade com o pipeline de processamento.

\begin{figure}[H]
    \centering
    \includegraphics[width=0.4\textwidth]{imagens/monocromo.jpg}
    \caption{Imagem do nó monocromo}
    \label{fig:interface-nodes-monocromo}
\end{figure}

\subsubsection{Brilho}

O sistema de ajuste de brilho segue uma abordagem similar ao nó monocromático, utilizando a classe base para definir a estrutura do nó e implementando a lógica específica na função \textit{run()}. O nó de ajuste de brilho inclui um parâmetro configurável que permite ao usuário definir o nível de brilho desejado da imagem, usando um \textit{slider} na interface gráfica.

O código abaixo demonstra a implementação do \textit{slider} que o nó de brilho utiliza para receber o valor do usuário:

\begin{lstlisting}[language=Python, caption={Implementação do slider para ajuste de brilho}, label={lst:slider-brilho}]
class BrightnessNode(NodeCore):
    def initialize(self, parent=None, node_tag: str | None = None, pos: list[int] | None = None):
        ...
        dpg.add_slider_int(
            tag=f"brightness_val_{idx}", 
            label="Value", 
            width=150, 
            min_value=0, 
            max_value=255, 
            default_value=0, 
            clamped=True, 
            callback=self.update_output
            )
\end{lstlisting}

A lógica de processamento do nó de brilho implementa a função \textit{run()}, que recebe a imagem de entrada e aplica o ajuste de brilho conforme o valor selecionado pelo usuário. Este sistema é integrado ao pipeline visual, permitindo que o nó de brilho seja conectado a outros nós para compor fluxos de edição personalizados.

\begin{figure}[H]
    \centering
    \includegraphics[width=0.6\textwidth]{imagens/brilho.jpg}
    \caption{Imagem do nó brilho}
    \label{fig:interface-nodes-brilho}
\end{figure}

\subsubsection{Desfoque}

O sistema de desfoque implementa alguns dos algoritmos mais comuns de desfoque, como o desfoque gaussiano e o desfoque de caixa. O nó de desfoque também inclui um parâmetro configurável que permite ao usuário final definir o nível de intensidade do desfoque e o método a ser aplicado.

O código abaixo demonstra a implementação do \textit{slider e o dropdown} que o nó de desfoque utiliza para receber o valor do usuário e o tipo:

\begin{lstlisting}[language=Python, caption={Implementação do slider e dropdown para desfoque}, label={lst:slider-desfoque}]
class BlurNode(NodeCore):
    BLUR_TYPES = {
        "BoxBlur": ImageFilter.BoxBlur,
        "GaussianBlur": ImageFilter.GaussianBlur}
    def initialize(self, parent=None, node_tag: str | None = None, pos: list[int] | None = None):
        ...
        dpg.add_combo(
            items=list(self.BLUR_TYPES.keys()),
            default_value="BoxBlur",
            tag=f"blur_type_{idx}",
            label="Blur Type",
            width=150,
            callback=self.update_output)
        dpg.add_slider_int(
            tag=f"blur_strength_{idx}",
            label="Strength",
            width=150,
            min_value=0,
            max_value=20,
            default_value=0,
            clamped=True,
            callback=self.update_output)
\end{lstlisting}

\begin{figure}[H]
    \centering
    \includegraphics[width=0.6\textwidth]{imagens/desfoque.jpg}
    \caption{Imagem do nó desfoque}
    \label{fig:interface-nodes-desfoque}
\end{figure}

\subsubsection{Efeitos Especiais}

Nós especializados como \textit{pixelation}, \textit{posterization} e \textit{dithering} implementam algoritmos específicos para criação de efeitos visuais. Cada nó mantém parâmetros configuráveis que permitem ajuste fino do efeito aplicado.

Os nós especializados utilizam a biblioteca ModernGL para implementar as edições especiais, aproveitando o poder de processamento direto na GPU.

A integração entre ModernGL e Dear PyGui permite a aplicação de efeitos visuais avançados em imagens, aproveitando o processamento gráfico acelerado por GPU. A criação de tais efeitos envolve a utilização, inicialmente, da biblioteca Pillow para manipulação básica da imagem, seguida pela conversão para uma textura, que então é enviada para a GPU utilizando o ModernGL. Em seguida são utilizados \textit{shaders} escritos em GLSL (OpenGL Shading Language). O \textit{vertex shader} prepara os vértices de um quadrilátero que cobre toda a área de renderização, enquanto o \textit{fragment shader} realiza a manipulação visual desejada, como pixelização, posterização ou dithering, processando cada pixel da imagem de acordo com o efeito programado.

Após o processamento, o resultado é renderizado em um \textit{framebuffer}. A imagem é lida de volta para a memória principal e convertida novamente em um objeto de imagem (PIL), com o efeito já aplicado. Por fim, a imagem é utilizada como textura na interface gráfica do Dear PyGui, permitindo sua visualização e manipulação pelo usuário dentro do editor baseado em nós.

A utilização de ModernGL e GLSL abstrai grande parte da complexidade do OpenGL tradicional, facilitando a implementação de novos efeitos visuais e a integração com o restante do sistema.

\subsubsection{Entrada}

O nó de entrada é responsável por carregar imagens do sistema de arquivos para o pipeline de processamento. Utiliza diálogo da biblioteca Dear PyGui para a seleção de arquivos.

O nó utiliza a biblioteca Pillow para abrir e converter imagens para o formato RGBA, garantindo compatibilidade com o restante do sistema de nós. Após o carregamento, a imagem é convertida em uma textura pela biblioteca NumPy, para que possa ser renderizada pela biblioteca Dear PyGui.

\begin{figure}[H]
    \centering
    \includegraphics[width=0.6\textwidth]{imagens/input.jpg}
    \caption{Imagem do nó de entrada}
    \label{fig:interface-nodes-entrada}
\end{figure}

\subsubsection{Saída}

O nó de saída é responsável por fornecer uma visualização do resultado final do pipeline de processamento e salvar a imagem processada de volta ao sistema de arquivos. O nó exibe a imagem resultante em um \textit{widget} de imagem dentro do nó.

O sistema de visualização utiliza texturas geradas pela biblioteca Dear PyGui para renderizar a imagem processada diretamente na interface gráfica. Isso permite ao usuário ver o resultado final do pipeline de edição em tempo real.

O nó de saída converte todas as modificações aplicadas ao longo do pipeline em uma imagem final, que pode ser salva em formatos comuns como PNG ou JPEG.

\begin{figure}[H]
    \centering
    \includegraphics[width=0.6\textwidth]{imagens/output.jpg}
    \caption{Imagem do nó de saída}
    \label{fig:interface-nodes-saida}
\end{figure}

\subsection{Posicionamento}

O programa implementa um sistema de posicionamento automático de nós que resolve um problema prático significativo na experiência do usuário. A função \textit{get\_available\_position()} pega a posição atual do canvas, baseada em onde o usuário deseja criar um novo nó e calcula uma posição livre próxima, evitando sobreposição com outros nós já existentes.

\subsection{Fluxo de Dados}

O sistema de conexões implementa um modelo de dataflow onde dados fluem unidirecionalmente através de conexões entre nós. Cada nó possui atributos de entrada e saída claramente definidos, utilizando o sistema de atributos do Dear PyGui para renderização e detecção de conexões.

As conexões são validadas em tempo real para garantir compatibilidade de tipos de dados. O sistema suporta múltiplas saídas de um nó para diferentes destinos, mas implementa validação para prevenir conexões circulares que resultariam em loops infinitos de processamento.

O fluxo de dados utiliza avaliação sob demanda (lazy evaluation), onde nós são executados apenas quando seus resultados são necessários. Esta abordagem otimiza a performance ao evitar cálculos desnecessários em ramos não utilizados do grafo.

A imagem [\ref{fig:interface-nodes-fluxo-completo}] ilustra um fluxo completo do sistema, demonstrando a interconexão entre diferentes tipos de nós e o caminho que os dados percorrem desde a entrada até a saída final.

\subsection{Sistema de colaboração}

A funcionalidade de colaboração em tempo real permite múltiplos usuários editarem o mesmo projeto simultaneamente. Mudanças são sincronizadas automaticamente entre todas as instâncias conectadas, utilizando um servidor WebSocket para comunicação eficiente.

Funcionalidades de sessão permitem usuários entrarem e saírem de sessões colaborativas sem interromper o fluxo de trabalho dos demais usuários.

\begin{figure}[H]
    \centering
    \includegraphics[width=0.8\textwidth]{imagens/websock.jpg}
    \caption{Imagem do servidor WebSocket}
    \label{fig:websocket-server}
\end{figure}

\subsection{Servidor Websocket}

O sistema de colaboração implementa um servidor WebSocket utilizando a biblioteca FastAPI, proporcionando comunicação bidirecional de baixa latência entre múltiplas instâncias do editor. Esta escolha tecnológica oferece performance superior a protocolos tradicionais HTTP para aplicações em tempo real.

\begin{lstlisting}[language=Python, caption={Servidor WebSocket para colaboração}, label={lst:websocket-server}]
# Servidor WebSocket para colaboracao
app = FastAPI()

@app.websocket("/ws")
async def websocket_endpoint(websocket: WebSocket):
    await websocket.accept()
    # Implementacao de comunicacao colaborativa
\end{lstlisting}

O servidor gerencia sessões de colaboração, autenticação de usuários e sincronização de estado entre clientes conectados. Implementa broadcasting eficiente de mudanças para todos os participantes de uma sessão.

\subsection{Cliente}

O cliente WebSocket integrado à aplicação principal estabelece conexões automáticas com o servidor e gerencia a sincronização de dados e o envio de dados ao servidor. O cliente implementa reconexão automática em caso de perda de conexão.

O sistema de polling integrado ao loop principal da aplicação garante um processamento contínuo de mensagens de rede sem bloquear a interface de usuário. Esse sistema garante que o usuário tenha as últimas atualizações em tempo real, mantendo a responsividade da aplicação. Esse sistema é implementado utilizando chamadas assíncronas que verificam periodicamente a fila de mensagens recebidas do servidor, processando-as conforme necessário.

O envio de informações ao servidor é realizado sempre que uma modificação é feita localmente, garantindo que todas as mudanças sejam refletidas na sessão colaborativa, imediatamente após a ação do usuário.

O cliente possui um sistema efetivo de gerenciamento de estado, garantindo que mudanças locais sejam refletidas corretamente no servidor e vice-versa. Implementa mecanismos de resolução de conflitos para lidar com edições simultâneas por múltiplos usuários, garantindo a integridade dos dados compartilhados.

\section{Resultados}

Na presente seção, serão apresentados os resultados obtidos a partir da implementação do sistema de edição de imagens baseado em nodes. Serão discutidos aspectos como performance, usabilidade e feedback dos usuários(maybe).

\begin{figure}[H]
    \centering
    \includegraphics[width=1\textwidth]{imagens/fluxo_completo.jpg}
    \caption{Fluxo completo do sistema}
    \label{fig:interface-nodes-fluxo-completo}
\end{figure}


% Capítulo 0X - Conclusão
\chapter{Conclusão}

Os objetivos propostos para este trabalho foram plenamente atingidos. O desenvolvimento de um aplicativo de edição de imagens baseado em programação visual por meio de grafos direcionados resultou em uma ferramenta intuitiva, acessível e eficiente. O projeto buscou simplificar o processo de edição para usuários sem conhecimento técnico, e isso foi concretizado por meio de uma interface gráfica amigável, módulos de edição facilmente conectáveis e uma curva de aprendizado reduzida. A arquitetura modular e o paradigma de fluxo de dados permitiram flexibilidade e experimentação, conforme planejado. A criação de um servidor dedicado para suportar a colaboração em tempo real também foi implementada com sucesso, atendendo à necessidade de trabalho conjunto entre múltiplos usuários.

Entre os resultados obtidos, destacam-se a implementação de funcionalidades essenciais de edição como, ajuste de brilho, desfoque e conversão para escala de cinza, além do suporte à colaboração em tempo real via WebSocket e um servidor dedicado. A performance do sistema se mostrou adequada para imagens de até um tamanho médio, e a interface visual baseada em nodes facilitou a compreensão do fluxo de edição, tornando o processo mais transparente para o usuário.

A principal contribuição do aplicativo está em sua abordagem inovadora, que agrega valor em relação aos softwares tradicionais de edição de imagens. Ao adotar o paradigma de programação visual, o sistema proporciona maior flexibilidade, modularidade e facilidade de uso, especialmente para iniciantes. A possibilidade de colaboração em tempo real também representa um diferencial relevante, ampliando o potencial de uso em ambientes educacionais e profissionais, para o ensino de artes visuais, design gráfico e outras áreas criativas.

Para trabalhos futuros, é recomendado a expansão do conjunto de módulos de edição fundamentais, a otimização do desempenho para imagens de alta resolução, a implementação de tutoriais interativos para facilitar ainda mais o aprendizado dos usuários, a integração com formatos de arquivo avançados, a ampliação das funcionalidades colaborativas, a possibilidade de criação de plugins personalizados e a adoção de técnicas de inteligência artificial para aprimorar as capacidades de edição automática, como recomendações de fluxos de edição, detecção de objetos e sugestões de melhorias.

Em suma, este trabalho revisitou os principais pontos abordados, reforçando que os objetivos foram alcançados e que as contribuições apresentadas representam avanços significativos no contexto de softwares de edição de imagens, abrindo portas ao mercado digital para novos concorrentes dos grande nomes já existentes, e oferecendo uma base sólida para futuras pesquisas e desenvolvimentos na área, assim promovendo a democratização do acesso a ferramentas de edição de imagens e diversificando as possibilidades de uso.



% ---
% Finaliza a parte no bookmark do PDF, para que se inicie o bookmark na raiz
% ---
\bookmarksetup{startatroot}% 
% ---

% ---------------------------------------------------------------------------------------------
% ---------------------------------------------------------------------------------------------
% ----------------------------------------------------------------------ELEMENTOS PÓS-TEXTUAIS-
% ---------------------------------------------------------------------------------------------
% ---------------------------------------------------------------------------------------------
\postextual

% ----------------------------------------------------------
% Referências bibliográficas
% ----------------------------------------------------------
\bibliographystyle{abntex2-alf}
\bibliography{bibliografia.bib}

% ---------------------------------------------------------------------------------------------
% ------------------------------------------------------Apêndices - Caso não existam, comentar-
% ---------------------------------------------------------------------------------------------
%  \begin{apendicesenv}

% % 	% Imprime uma página indicando o início dos apêndices
%  	\partapendices

% % % ----------------------------------------------------------
%  \chapter{Quisque libero justo}
%  \lipsum[1-3]
% % % ----------------------------------------------------------

%  % Lorem ipsum dolor sit amet, consectetur adipiscing elit. Maecenas laoreet porttitor dui sit amet tempus. Phasellus finibus ac tortor eget lobortis. Suspendisse et nunc velit. Vestibulum vulputate, urna laoreet hendrerit dignissim, urna massa facilisis ante, quis vestibulum erat elit blandit eros. Cras a justo eu risus lobortis lobortis vitae luctus mauris. Nunc congue consequat justo, eget imperdiet magna porttitor nec. Class aptent taciti sociosqu ad litora torquent per conubia nostra, per inceptos himenaeos. Vestibulum lorem ex, vulputate nec nibh ac, consequat vulputate ex. Aliquam dignissim dapibus orci, eget porttitor dui rutrum a. Pellentesque lacinia ante in sapien lacinia, a ultrices libero faucibus. In erat purus, elementum ut lectus nec, iaculis accumsan mi. Donec bibendum ante ac nulla hendrerit consectetur. Aliquam a mauris ultricies, ornare lacus sed, convallis orci. Class aptent taciti sociosqu ad litora torquent per conubia nostra, per inceptos himenaeos.

% % Etiam viverra risus non sodales vehicula. In convallis sapien vel neque aliquam, non porttitor turpis dapibus. Suspendisse consectetur volutpat purus, quis ultrices odio tempor id. Duis laoreet justo quam, vitae ultrices tellus dapibus vel. Mauris placerat ipsum finibus odio luctus suscipit. Duis facilisis non leo non varius. Sed et magna feugiat, rutrum diam vel, maximus nunc.

% % Donec vehicula dapibus pharetra. Etiam et libero pharetra nulla ultricies laoreet sit amet vel sapien. Proin ultrices vitae nibh quis accumsan. Nullam in tortor erat. Quisque imperdiet dui venenatis pulvinar laoreet. Nulla ut nunc sed urna consectetur finibus id ut justo. Sed in placerat massa, quis tempor orci.

% % Nulla in nisi congue, blandit erat in, viverra quam. Proin eget tincidunt turpis. Phasellus sit amet neque mauris. Praesent nec dui pretium, viverra tellus tempor, gravida sapien. Donec fermentum pulvinar libero vel vestibulum. Suspendisse vitae iaculis diam. Mauris lobortis volutpat turpis eu lobortis.

% % % ----------------------------------------------------------
%  \chapter{Nullam elementum urna vel imperdiet sodales elit ipsum pharetra ligula
%  ac pretium ante justo a nulla curabitur tristique arcu eu metus}
%  \lipsum[1-5]
% % % ----------------------------------------------------------
% % Lorem ipsum dolor sit amet, consectetur adipiscing elit. Maecenas laoreet porttitor dui sit amet tempus. Phasellus finibus ac tortor eget lobortis. Suspendisse et nunc velit. Vestibulum vulputate, urna laoreet hendrerit dignissim, urna massa facilisis ante, quis vestibulum erat elit blandit eros. Cras a justo eu risus lobortis lobortis vitae luctus mauris. Nunc congue consequat justo, eget imperdiet magna porttitor nec. Class aptent taciti sociosqu ad litora torquent per conubia nostra, per inceptos himenaeos. Vestibulum lorem ex, vulputate nec nibh ac, consequat vulputate ex. Aliquam dignissim dapibus orci, eget porttitor dui rutrum a. Pellentesque lacinia ante in sapien lacinia, a ultrices libero faucibus. In erat purus, elementum ut lectus nec, iaculis accumsan mi. Donec bibendum ante ac nulla hendrerit consectetur. Aliquam a mauris ultricies, ornare lacus sed, convallis orci. Class aptent taciti sociosqu ad litora torquent per conubia nostra, per inceptos himenaeos.

% % Etiam viverra risus non sodales vehicula. In convallis sapien vel neque aliquam, non porttitor turpis dapibus. Suspendisse consectetur volutpat purus, quis ultrices odio tempor id. Duis laoreet justo quam, vitae ultrices tellus dapibus vel. Mauris placerat ipsum finibus odio luctus suscipit. Duis facilisis non leo non varius. Sed et magna feugiat, rutrum diam vel, maximus nunc.

% % Donec vehicula dapibus pharetra. Etiam et libero pharetra nulla ultricies laoreet sit amet vel sapien. Proin ultrices vitae nibh quis accumsan. Nullam in tortor erat. Quisque imperdiet dui venenatis pulvinar laoreet. Nulla ut nunc sed urna consectetur finibus id ut justo. Sed in placerat massa, quis tempor orci.

% % Nulla in nisi congue, blandit erat in, viverra quam. Proin eget tincidunt turpis. Phasellus sit amet neque mauris. Praesent nec dui pretium, viverra tellus tempor, gravida sapien. Donec fermentum pulvinar libero vel vestibulum. Suspendisse vitae iaculis diam. Mauris lobortis volutpat turpis eu lobortis.

% \end{apendicesenv}

% ---------------------------------------------------------------------------------------------
% ---------------------------------------------------------Anexos - Caso não existam, comentar-
% ---------------------------------------------------------------------------------------------
\include{99_anexos}

%---------------------------------------------------------------------
% INDICE REMISSIVO
%---------------------------------------------------------------------
%\printindex

\end{document}

