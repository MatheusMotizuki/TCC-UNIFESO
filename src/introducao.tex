\chapter[Introdução]{Introdução}

% Introdução !!!!

Os aplicativos de edição de imagens têm uma longa trajetória. Um dos marcos iniciais que viria a se tornar o Adobe Photoshop, criado pelos irmãos John e Thomas Knoll em 1987. Após um período de testes e pequenas distribuições sob o nome de Barneyscan XP, em 1989, o programa chamou a atenção da Adobe Systems, que fez um contrato de licenciamento com os irmãos Knoll. Assim, em 1990, foi lançado oficialmente o Adobe Photoshop 1.0, inicialmente para o sistema Macintosh \cite{BarneyScan}.

Com o avanço das tecnologias de hardware e software, surgiram novos programas de edição de imagem, estabelecendo padrões que hoje são referências na indústria do design gráfico. Ferramentas como o próprio Photoshop, que desde a sua concepção vêm se adaptando aos padrões de cada época, e o Gimp (GNU Image Manipulation Program) tornaram-se amplamente reconhecidos \cite{PhotoEditors}.

Ao longo do tempo, os softwares de edição de imagem buscaram tornar suas interfaces mais intuitivas e simplificadas, a fim de atingir um público mais amplo e não técnico. Essa tendência de design, proporcionou a criação de diversos aplicativos, dentro e fora da área de edição de imagens, mais simples e com um menor requerimento de expertise em uma determinada área.

Um conceito que se destacou durante esse processo foi o desenvolvimento de interfaces visuais, ou programação visual. Isso ocorre porque, de maneira geral, a compreensão de informações apresentadas graficamente tendem a ser mais intuitivas. Essa abordagem reforça a importância de interfaces gráficas bem planejadas, permitindo com que os usuários apliquem seus conhecimentos de forma mais prática, otimizando o fluxo de trabalho e a produtividade.

A Programação de Fluxo de Dados (PFD), ou \textit{Dataflow Programming (DFP)}, é um paradigma de programação que modela a lógica de um programa como um grafo dirigido, de forma semelhante aos Diagramas de Fluxo de Dados (DFD). Nesse modelo, o programa é representado como um conjunto de nós, também conhecido como vértices interligados por conexões direcionadas, chamadas de arcos (ou bordas), que indicam o percurso dos dados entre as operações. Cada nó representa uma função ou operação específica \cite{dataflow-procedure, 48862}.

A programação de fluxo de dados permite, por natureza, um alto grau de paralelismo. Diferentemente da arquitetura tradicional de von Neumann, onde as instruções são executadas em ordem e o controle de fluxo é centralizado, na PFD a execução ocorre de maneira assíncrona e descentralizada. Cada \textit{node} (nó) é executado assim que todos os dados necessários para sua operação estejam disponíveis. Esse modelo elimina a necessidade de um controle centralizado de fluxo, permitindo que múltiplas operações sejam realizadas simultaneamente, com melhor aproveitamento dos recucos computacionais disponíveis \cite{48862}.

O alto paralelismo natural da PFD faz com que, na área de edição de imagem, o fluxo de edição se torne descentralizado e não linear. Em programas tradicionais de edição de imagem, o usuário segue uma ordem linear de operações, fazendo com que cada mudança dependa da anterior. Com a implementação de um modelo Dataflow em um programa de edição de imagem, esse foco muda e se torna descentralizado, permitindo que o usuário edite uma imagem de diversas maneiras, sem depender de uma sequência rígida.

Programas desenvolvidos sob este paradigma são estruturados por meio de grupos de nodes (ou blocos), interligados por conexões que representam o fluxo de dados. Cada node possui portas de entrada e/ou saída, que pode exercer diferentes funções dentro do sistema, demonstrado na figura ~\ref{fig:dataflow} De forma geral, esses nós são classificados em três categorias principais, sendo elas: blocos de entrada, blocos de processamento e blocos de saída. As informações percorrem o programa seguindo essas conexões, introduzidos pelos blocos de entrada, sendo transformados conforme passam pelos blocos processadores, até atingirem os destinos finais nos blocos de saída, que coletam os resultados \cite{dataflow}.

\begin{figure}[ht]
    \centering
    \includegraphics[width=1\textwidth]{imagens/dfpex.png}
    \caption{
    Um simples programa (a) e o seu equivalente em no formato Dataflow (b)}
    \label{fig:dataflow}
    \cite{dataflow}
\end{figure}

\section{Justificativa}

Este trabalho propõe a criação de um aplicativo de edição de imagens baseado em grafos para a modelagem do fluxo da edição, tornando possível não apenas compreender e documentar o processo, mas também oferecer ao usuário final, que não possui o conhecimento técnico necessário para operar um aplicativo de edição mais avançado, meios mais intuitivos de modificar e experimentar sequências de transformações. Dessa forma, obtém-se uma solução de edição descentralizada, em que a criatividade é o único limite para a edição. Além disso, a adoção dessa metodologia pode servir como base para o desenvolvimento de aplicações que automatizem ou recomendem etapas de edição.

\section{Motivação}

A criação deste aplicativo origina-se principalmente na dificuldade de utilização de softwares profissionais, que, apesar de sua popularização, ainda apresentam uma barreira de entrada significativa para usuários não técnicos. Esses softwares tendem a não ser amigáveis para novos interessados na área, exigindo muitas vezes um verdadeiro “curso” para que se aprenda a utilizá-los, se tornando mais um requerimento do que apenas um complemento para usuários já familiarizados que querem elevar o nível de conhecimento.

\section{Objetivos}

Na presente seção serão expostos os objetivos do trabalho, tanto gerais quanto específicos.

\subsection{Objetivos Gerais}

O objetivo principal deste projeto é a criação de um aplicativo de edição de foto baseado em nodes, com a sua principal finalidade sendo a implementação do diagrama de fluxo de dados, de uma forma visual, de forma que o usuário possa editar fotos e/ou imagens de uma forma disruptiva e não linear.

Diferindo da tradicional forma de edição fotográfica sendo, uma forma linear de editar as imagens, com instruções uma após a outra, assim limitando o usuário a seguir um padrão ao invés de dar a liberdade para que o usuário possa, da forma que preferir, editar a imagem.

\subsection{Objetivos Específicos}

\begin{itemize}
    \item Implementar uma UI interativa e de fácil compreensão, que facilite a manipulação e configuração de componentes, sem a necessidade de grande um conhecimento técnico.
    \item Implementar o paradigma de programação de fluxo de dados, permitindo que usuários construam pipelines de processamento de forma visual e modular.
    \item Implementar criação de Shaders para edições de imagens e/ou criação de imagens dinâmicas.
    \item Implementar \textit{nodes} funcionais capazes de realizar mudanças nas imagens, atendendo os parâmetros e requisitos definidos pelo usuário, independentemente da posição ou ordem no grafo.
    \item Implementar \textit{nodes} capazes de criar Shaders, tanto em versões de código, feitos pelo usuário, quanto nodes de código já previamente feitos pelo programa.
    \item Implementar um servidor websocket para sessões colaborativas, em que dois ou mais usuários possam editar a mesma imagem ao mesmo tempo.
\end{itemize}

\section{Organização dos Capítulos}

Na presente seção será apresentada a organização deste trabalho, divido em seções, iniciando pela introdução, seguida pela fundamentação teórica, trabalhos relacionados, desenvolvimento e conclusão do trabalho.