\begin{otherlanguage*}{english}
    \vspace{\onelineskip}
    Image editing applications have existed for a long time and, since their inception, have continued to present usability challenges for users without technical knowledge. The objective of this work is to develop a simplified and more straightforward image editing application using the dataflow programming paradigm, where data flows between nodes through edges that are interconnected in a directed graph. To achieve this, three distinct software solutions were created in order to compare different programming languages and their capabilities in implementing the dataflow programming paradigm. The creation of such an application brings significant advantages, such as reducing the learning curve for new users and enabling individuals without technical experience to effectively edit images. Furthermore, the use of the dataflow programming paradigm provides a modular and flexible approach to image editing, facilitating user experimentation and creativity. This creates new possibilities for the field of image editing, democratizing access to advanced tools.

    \noindent
\textbf{Keywords}: Dataflow Programming, Directed Graph, Image Editing, Usability, Practicality.
\end{otherlanguage*}