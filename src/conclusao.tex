\chapter{Conclusão}

Os objetivos propostos para este trabalho foram plenamente atingidos. O desenvolvimento de um aplicativo de edição de imagens baseado em programação visual por meio de grafos direcionados resultou em uma ferramenta intuitiva, acessível e eficiente. O projeto buscou simplificar o processo de edição para usuários sem conhecimento técnico, e isso foi concretizado por meio de uma interface gráfica amigável, módulos de edição facilmente conectáveis e uma curva de aprendizado reduzida. A arquitetura modular e o paradigma de fluxo de dados permitiram flexibilidade e experimentação, conforme planejado. A criação de um servidor dedicado para suportar a colaboração em tempo real também foi implementada com sucesso, atendendo à necessidade de trabalho conjunto entre múltiplos usuários.

Entre os resultados obtidos, destacam-se a implementação de funcionalidades essenciais de edição como, ajuste de brilho, desfoque e conversão para escala de cinza, além do suporte à colaboração em tempo real via WebSocket e um servidor dedicado. A performance do sistema se mostrou adequada para imagens de até um tamanho médio, e a interface visual baseada em nodes facilitou a compreensão do fluxo de edição, tornando o processo mais transparente para o usuário.

A principal contribuição do aplicativo está em sua abordagem inovadora, que agrega valor em relação aos softwares tradicionais de edição de imagens. Ao adotar o paradigma de programação visual, o sistema proporciona maior flexibilidade, modularidade e facilidade de uso, especialmente para iniciantes. A possibilidade de colaboração em tempo real também representa um diferencial relevante, ampliando o potencial de uso em ambientes educacionais e profissionais, para o ensino de artes visuais, design gráfico e outras áreas criativas.

Para trabalhos futuros, é recomendado a expansão do conjunto de módulos de edição fundamentais, a otimização do desempenho para imagens de alta resolução, a implementação de tutoriais interativos para facilitar ainda mais o aprendizado dos usuários, a integração com formatos de arquivo avançados, a ampliação das funcionalidades colaborativas, a possibilidade de criação de plugins personalizados e a adoção de técnicas de inteligência artificial para aprimorar as capacidades de edição automática, como recomendações de fluxos de edição, detecção de objetos e sugestões de melhorias.

Em suma, este trabalho revisitou os principais pontos abordados, reforçando que os objetivos foram alcançados e que as contribuições apresentadas representam avanços significativos no contexto de softwares de edição de imagens, abrindo portas ao mercado digital para novos concorrentes dos grande nomes já existentes, e oferecendo uma base sólida para futuras pesquisas e desenvolvimentos na área, assim promovendo a democratização do acesso a ferramentas de edição de imagens e diversificando as possibilidades de uso.