\chapter[Trabalhos Relacionados]{Trabalhos Relacionados}

Na presente seção serão apresentados softwares de edição de imagem baseados em nodes (ou grafos). Serão analisados os principais aspectos desses programas, destacando como cada um implementa a modelagem de fluxo de edição, quais recursos são oferecidos aos usuários e a praticidade de uso. O objetivo é identificar pontos fortes e limitações, de modo a fornecer uma base comparativa para a proposta desenvolvida neste trabalho.

\section{Gimel Studio}

O Gimel Studio é um software de código aberto destinado à edição de imagens baseado em nodes. Desenvolvido na linguagem de programação Dart e utilizando o framework Flutter para a construção da interface gráfica, o Gimel Studio apresenta similaridades visuais com editores tradicionais, como o Adobe Photoshop. A Figura~[\ref{fig:gimelstudio}] apresenta a interface do Gimel Studio, que é caracterizada por sua semelhança com editores convencionais, facilitando a adaptação de usuários familiarizados com esse estilo de interface.

Atualmente o Gimel Studio se encontra em estágio inicial de desenvolvimento. O repositório oficial fornece instruções detalhadas para a execução do projeto localmente. Até o presente momento, existe somente um lançamento de uma versão alpha, feita em Janeiro de 2022. Desde então, o desenvolvimento não apresenta atualizações significativas.

O Gimel Studio oferece um conjunto básico de nodes/módulos de edição, incluindo ajustes de brilho, contraste, saturação, desfoque e conversão para escala de cinza. A interface do usuário é intuitiva, permitindo a criação e conexão de nodes de forma visual. No entanto, o software não possui funcionalidades avançadas e uma comunidade pouco ativa, o que limita seu potencial de crescimento e adoção.

Em comparação com o aplicativo proposto neste trabalho, o Gimel Studio apresenta uma interface mais tradicional, semelhante a editores de imagem convencionais, o que pode ser vantajoso para usuários acostumados a esse estilo. No entanto, o conjunto limitado de funcionalidades e a falta de suporte à colaboração em tempo real representam desvantagens significativas. O aplicativo desenvolvido neste trabalho busca oferecer uma experiência mais robusta e colaborativa, com uma gama maior de módulos de edição e uma interface mais refinada, visando atender a um público mais amplo, incluindo usuários sem conhecimento técnico avançado.

\begin{figure}[H]
    \centering
    \includegraphics[width=1\textwidth]{imagens/gimelstudio.png}
    \caption{Imagem do Gimel Studio}
    \label{fig:gimelstudio}
    \cite{gimelstudio}
\end{figure}


\section{Cresliant}

O Cresliant é um software de código aberto destinado à edição de imagens baseado em nodes. Desenvolvido na linguagem de programação Python, o Cresliant destaca-se por oferecer uma interface gráfica intuitiva e acessível ao usuário final, frequentemente referida como \textit{User-Friendly}. Entre as tecnologias utilizadas em sua criação, destaca-se o uso do \textit{Poetry}, para o gerenciamento eficiente de dependências e ambientes virtuais, bem como a biblioteca PILLOW (fork da Python Imaging Library - PIL), que desempenha papel crucial no processamento e manipulação de imagens do software. A Figura~[\ref{fig:cresliant}] apresenta a interface do Cresliant, que é caracterizada por sua simplicidade e facilidade de uso.

Apesar de o repositório do Cresliant, no Github, não apresentar atualizações recentes, sendo o último \textit{commit} registrado há aproximadamente dois anos, e a versão mais recente datada de dezembro de 2023, o software permanece funcional e estável, cumprindo de maneira eficaz o propósito a que se destina.

O Cresliant oferece, por padrão, poucos nodes/módulos de edição. No entanto, possibilita que o usuário crie novos nodes/módulos e os adicione como submódulos/\textit{addons} externos, sem a necessidade de modificações diretas no código-fonte. Essa abordagem promove uma certa liberdade ao usuário final, mas ao mesmo tempo limita a funcionalidade apenas àqueles que possuem conhecimento 
técnico suficiente para realizar tal integração de novas funcionalidades.

O aplicativo Cresliant foca em fornecer uma experiência de edição de imagens baseada em nodes, priorizando a simplicidade e a facilidade de uso. Em comparação com o aplicativo proposto neste trabalho, o Cresliant apresenta uma interface mais básica e um conjunto limitado de funcionalidades nativas. O aplicativo desenvolvido neste trabalho busca oferecer uma experiência mais robusta, com uma gama maior de módulos de edição e uma interface mais refinada, visando atender a um público mais amplo, incluindo usuários sem conhecimento técnico avançado, além de oferecer uma funcionalidade única, sendo a colaboração em tempo real entre múltiplos usuários, o que não está presente no Cresliant.

\begin{figure}[H]
    \centering
    \includegraphics[width=1\textwidth]{imagens/cresliant.png}
    \caption{Imagem do Cresliant}
    \label{fig:cresliant}
    \cite{cresliant}
\end{figure}


\section{PixiEditor}

O PixiEditor é software de código aberto destinado à criação de gráficos procedurais por meio de nodes. Desenvolvido na linguagem de programação Csharp, o PixiEditor destaca-se por sua robustez e ampla gama de funcionalidades. Além do editor de gráficos procedurais, o software oferece recursos adicionais, como criação de animações 2D, edição tradicional de imagens 2D, entre outras ferramentas, sendo ainda compatível com múltiplas plataformas. A Figura~[\ref{fig:pixieditor}] apresenta a interface do PixiEditor, que é caracterizada por sua complexidade e riqueza de recursos.

Atualmente o PixiEditor encontra-se com o desenvolvimento ativo, com sua ultima versão lançada no dia 10 de setembro de 2025. Com sua vasta gama de funcionalidades, o PixiEditor se apresenta como uma alternativa viável para profissionais e entusiastas que buscam uma solução completa para criação e edição gráfica.

Em comparação com o aplicativo proposto neste trabalho, o PixiEditor oferece uma gama mais ampla de funcionalidades, indo além da edição de imagens baseada em nodes. A interface do PixiEditor é mais complexa, refletindo a diversidade de recursos oferecidos, o que pode representar um desafio para usuários iniciantes. Em contrapartida, o aplicativo proposto neste trabalho prioriza a simplicidade e a facilidade de uso, visando atender especialmente usuários sem conhecimento técnico avançado, e se concentra exclusivamente na edição de imagens por meio de nodes, proporcionando uma experiência mais direta e acessível.

\begin{figure}[H]
    \centering
    \includegraphics[width=1\textwidth]{imagens/pixi.png}
    \caption{Imagem do Pixi Editor}
    \label{fig:pixieditor}
    \cite{pixieditor}
\end{figure}


\section{Photoshop}

O Adobe Photoshop é um dos mais poderosos e populares editores de imagens e fotos do mundo, com diversas ferramentas para edição, criação e manipulação de imagens. 
Desenvolvido pela Adobe, é amplamente utilizado por profissionais de design gráfico, fotografia, ilustração digital e outras áreas relacionadas. A Figura~[\ref{fig:photoshop}] apresenta a interface do Photoshop, que é caracterizada por sua complexidade e riqueza de recursos.

O Photoshop oferece uma ampla gama de recursos, incluindo ferramentas de seleção, camadas, máscaras, filtros, ajustes de cor, suporte a plugins, automação por scripts, 
integração com outros softwares da Adobe (como Illustrator e After Effects), além de suporte a múltiplos formatos de arquivo.

O Photoshop não é baseado em nodes, mas sim em uma interface tradicional de edição de imagens. 
Apesar disso, é um software amplamente reconhecido e utilizado na indústria de design gráfico e edição de imagens, sendo referência em qualidade, flexibilidade e produtividade.

O software é proprietário, com licença comercial, disponível para Windows e macOS. Possui uma comunidade global, extensa documentação, suporte técnico oficial e uma vasta gama de plugins e integrações. 
O fluxo de trabalho é altamente profissional, com recursos avançados de automação e personalização, indicado para usuários avançados e profissionais.

Em contraste com o aplicativo proposto neste trabalho, o Photoshop oferece uma gama muito mais ampla de funcionalidades e recursos avançados, atendendo a um público profissional. O aplicativo desenvolvido neste trabalho foca na simplicidade e acessibilidade, visando atender usuários sem conhecimento técnico avançado, com ênfase na edição de imagens por meio de nodes e colaboração em tempo real, características que não estão presentes no Photoshop.

\begin{figure}[H]
    \centering
    \includegraphics[width=1\textwidth]{imagens/photoshop.jpg}
    \caption{Imagem do Photoshop}
    \label{fig:photoshop}
    \cite{photoshop}
\end{figure}


\section{MS Paint}

O MS Paint é um editor e criador de imagens e desenhos tradicional, integrado ao sistema operacional Windows desde suas primeiras versões. É um software simples, gratuito e voltado para tarefas básicas de edição e criação de imagens. A Figura~[\ref{fig:mspaint}] apresenta a interface do MS Paint, que é caracterizada por sua simplicidade e facilidade de uso.

O MS Paint oferece ferramentas elementares como pincel, lápis, balde de tinta, formas geométricas, seleção e recorte, sendo indicado para usuários iniciantes ou para edições rápidas e descomplicadas.

O MS Paint não possui suporte a nodes, plugins ou extensões, e sua interface é extremamente simples, sem recursos avançados de edição. Não há integração com outros softwares, mas sua facilidade de uso e disponibilidade o tornam uma ferramenta acessível para o público geral, sendo frequentemente utilizado para tarefas cotidianas, anotações rápidas e ilustrações simples.

O MS Paint é uma simples ferramenta de edição de imagens, focada em tarefas básicas e acessibilidade. Em contraste, o aplicativo proposto neste trabalho oferece uma abordagem mais avançada e flexível para edição de imagens, utilizando nodes e permitindo colaboração em tempo real, características que não estão presentes no MS Paint. O aplicativo desenvolvido visa atender a um público mais amplo, incluindo usuários sem conhecimento técnico avançado, mas com necessidades de edição mais complexas do que aquelas atendidas pelo MS Paint.

\begin{figure}[H]
    \centering
    \includegraphics[width=1\textwidth]{imagens/paint.jpg}
    \caption{Imagem do MS Paint \\ (Fonte: autoria própria)}
    \label{fig:mspaint}
\end{figure}


\section{Comparativo}

A Tabela [\ref{tab:comparativo}] apresenta um comparativo entre os softwares analisados nesta seção, destacando suas principais características, funcionalidades e limitações.

\begin{table}[H]
    \centering
    \begin{tabular}{|l|c|c|c|c|}
        \hline
        \textbf{Software} & \textbf{Baseado em Nodes} & \textbf{Funcionalidades} & \textbf{Web} & \textbf{Colaboração} \\
        \hline
        Gimel
        Studio & Sim & Básico & Não & Não \\
        \hline
        Cresliant & Sim & Básico & Não & Não \\
        \hline
        Pixi
        Editor & Sim & Avançado & Não & Não \\
        \hline
        Adobe
        Photoshop & Não & Muito Avançado & Não & Não \\
        \hline
        MS
        Paint & Não & Muito Básico & Não & Não\\
        \hline
        Aplicativo
        Proposto & Sim & Básico & Sim & Sim \\
        \hline
    \end{tabular}
    \caption{Comparativo entre softwares de edição de imagem}
    \label{tab:comparativo}
\end{table}