\chapter[Relacionados]{Trabalhos Relacionados}

Na presente seção serão apresentados softwares de edição de imagem baseados em nodes (ou grafos). Serão analisados os principais
aspectos desses programas, destacando como cada um implementa a modelagem de fluxo de edição, quais recursos são oferecidos aos 
usuários e a praticidade de uso. O objetivo é identificar pontos fortes e limitações, de modo a fornecer uma base comparativa 
para a proposta desenvolvida neste trabalho.

\section{Gimel Studio}

O Gimel Studio é um software de código aberto destinado à edição de imagens baseado em nodes. Desenvolvido na linguagem de 
programação Dart e utilizando o framework Flutter para a construção da interface gráfica, o Gimel Studio apresenta similaridades
visuais com editores tradicionais, como o Adobe Photoshop.

Atualmente o Gimel Studio se encontra em estágio inicial de desenvolvimento. O repositório oficial fornece instruções detalhadas para
a execução do projeto localmente. Até o presente momento, existe somente um lançamento de uma versão alpha, feita em Janeiro de 2022.
Desde então, o desenvolvimento não apresenta atualizações significativas.

% falar das funcionalidades do gimel studio

\begin{figure}[ht]
    \centering
    \includegraphics[width=1\textwidth]{imagens/gimelstudio.png}
    \caption{Imagem do Gimel Studio}
    \label{fig:gimelstudio}
    \cite{gimelstudio}
\end{figure}

\section{Cresliant}

O Cresliant é software de código aberto destinado à edição de imagens baseado em nodes. Desenvolvido na linguagem de programação Python,
o Cresliant destaca-se por oferecer uma interface gráfia intuitiva e acessível ao usuário final, frequentemente referida como
\textit{User-Friendly}. Entre tecnologias utilizadas em sua criação, destaca-se o uso do \textit{Poetry}, para o gerenciamento 
eficiente de dependências e ambientes virtuais, bem como a biblioteca PILLOW (fork da Python Imaging Library - PIL), 
que faz o papel crucial no processamento e manipulação de imagens do software.

Apesar de o repositório do Cresliant, no Github, não apresentar atualizações recentes, sendo o último \textit{commit} registrado há
aproximadamente dois anos, e a versão mais recente datada de dezembro de 2023, o software permanece funcional e estável, cumprindo 
de maneira eficaz o propósito a que se destina.

O Cresliant oferece, por padrão, poucos nodes/módulos de edição. No entanto, possibilita que o usuário crie novos nodes/módulos 
e os adicione como submódulos/\textit{addons} externos, sem a necessidade de modificações diretas no código-fonte. Essa abordagem 
promove uma certa liberdade ao usuário final, mas ao mesmo tempo limita a funcionalidade apenas àqueles que possuem conhecimento 
técnico suficiente para realizar tal integração de novas funcionalidades.

\begin{figure}[ht]
    \centering
    \includegraphics[width=1\textwidth]{imagens/cresliant.png}
    \caption{Imagem do Cresliant}
    \label{fig:cresliant}
    \cite{cresliant}
\end{figure}

\section{PixiEditor}

O PixiEditor é software de código aberto destinado à criação de gráficos procedurais por meio de nodes. Desenvolvido na linguagem de programação
Csharp, o PixiEditor destaca-se por sua robustez e ampla gama de funcionalidades. Além do editor de gráficos procedurais, o software
oferece recursos adicionais, como criação de animações 2D, edição tradicional de imagens 2D, entre outras ferramentas, sendo ainda compatível com múltiplas plataformas.

Atualmente o PixiEstudio encontra-se com o desenvolvimento ativo, com sua ultima versão lançada no dia 10 de setembro de 2025.

% falar das funcionalidades

\begin{figure}[ht]
    \centering
    \includegraphics[width=1\textwidth]{imagens/pixi.png}
    \caption{Imagem do Pixi Editor}
    \label{fig:pixieditor}
    \cite{pixieditor}
\end{figure}

\section{Photoshop}

Photoshop é um editor de imagens tradicional.

\section{MS Paint}

MS Paint é um editor/criador de imagens e/ou desenhos tradicional, funcionando mais como um canva para desenhos.