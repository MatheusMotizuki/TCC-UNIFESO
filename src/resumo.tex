\vspace{\onelineskip}

Ferramentas de edição de imagem existem há muito tempo e, desde a sua concepção, esses softwares sempre apresentaram dificuldades de utilização para usuários sem conhecimento técnico. A proposta deste trabalho é desenvolver um aplicativo de edição de imagens simplificado e mais direto. A criação de tal aplicativo oferece vantagens significativas, como a redução da curva de aprendizado para novos usuários, permitindo que indivíduos sem experiência técnica possam editar imagens de forma eficaz. Além disso, a utilização do paradigma de programação de fluxo de dados (\textit{Dataflow Programming}) proporciona uma abordagem modular e flexível para a edição de imagens, facilitando a experimentação e a criatividade dos usuários.

\noindent 

\textbf{Palavras-chave}: Programação visual, Edição de imagens, Grafo direcionado.