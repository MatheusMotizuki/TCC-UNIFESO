\vspace{\onelineskip}

Aplicativos de edição de imagens existem há muito tempo e, desde a sua criação, não deixam de apresentar desafios de usabilidade para usuários sem conhecimento técnico. O objetivo deste trabalho é desenvolver um aplicativo de edição de imagens simplificado e mais direto, utilizando o paradigma de programação de fluxo de dados, onde dados fluem entre nós através por arestas que estão interconectadas em um grafo direcionado, para isso, três soluções de software distintas foram criadas, a fim de comparar diferentes linguagens de programação e suas capacidades na implementação do paradigma de programação de fluxo de dados. A criação de tal aplicativo traz vantagens significativas, como a redução da curva de aprendizado para novos usuários e a possibilidade de indivíduos sem experiência técnica editarem imagens de forma eficaz. Além disso, o uso do paradigma de programação de fluxo de dados proporciona uma abordagem modular e flexível para a edição de imagens, facilitando a experimentação e criatividade dos usuários. Criando assim novas possibilidades para o campo de edição de imagens, democratizando o acesso a ferramentas avançadas.

\noindent 

\textbf{Palavras-chave}: Programação de fluxo de dados, Grafo Direcionado, Edição de imagens, Usabilidade, Praticidade.
