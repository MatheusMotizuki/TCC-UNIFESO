% ---
% Inserir a ficha bibliografica
% ---
% A biblioteca da universidade lhe fornecerá um PDF
% com a ficha catalográfica definitiva após a defesa do trabalho. Quando estiver
% com o documento, salve-o como PDF no diretório do seu projeto e substitua todo
% o conteúdo de implementação deste arquivo pelo comando abaixo:

\begin{fichacatalografica}
    %\includepdf{fichacatalografica.pdf}
\end{fichacatalografica}


\newpage
% ---
% ---


% Isto é um exemplo de Folha de aprovação, elemento obrigatório da NBR
% 14724/2011 (seção 4.2.1.3). Você pode utilizar este modelo até a aprovação
% do trabalho. Após isso, substitua todo o conteúdo deste arquivo por uma
% imagem da página assinada pela banca com o comando abaixo:
%
% \includepdf{folhadeaprovacao_final.pdf}
%
\begin{folhadeaprovacao}

  \begin{center}
    {\ABNTEXchapterfont\large\imprimirautor}

    \vspace*{\fill}\vspace*{\fill}
    {\ABNTEXchapterfont\bfseries\Large\imprimirtitulo}
    \vspace*{\fill}
    
    \hspace{.45\textwidth}
    \begin{minipage}{.5\textwidth}
        \imprimirpreambulo
    \end{minipage}%
    \vspace*{\fill}
   \end{center}
 

    %\todo[inline]{Alterar o texto, talvez criar alguma macro condicional}
    \begin{center}
        \noindent
        Trabalho de Conclusão de Curso apresentado ao Centro Universitário Serra dos Órgãos como requisito obrigatório para obtenção do título de Bacharel em Ciência da Computação.
    \end{center}
   % Caso prefira fixar a data, basta mudar o comando \today pela data,
   %   como no exemplo abaixo:
   %Trabalho aprovado. \imprimirlocal, 7 de setembro de 1822:

    
   \assinatura{\textbf{\imprimirorientador} \\ Orientador} 
   \assinatura{\textbf{Professor} \\ Nome Membro 1}     % título e nome
   \assinatura{\textbf{Doutor} \\ Nome membro 2}        % título e nome
   %\assinatura{\textbf{Professor} \\ Convidado 3}
   %\assinatura{\textbf{Professor} \\ Convidado 4}
      
   \begin{center}
    \vspace*{0.5cm}
    {\large\imprimirlocal}
    \par
    {\large\imprimirdata}
    \vspace*{1cm}
  \end{center}
  
\end{folhadeaprovacao}

% ---------------------------------------------------------------------------------------------
% ---------------------------------------------------------------------------------Dedicatória-
% ---------------------------------------------------------------------------------------------
\begin{dedicatoria}
	\vspace*{\fill}
        \flushright
	% \centering
	% \noindent
	\textit{Dedico este trabalho ao meu falecido pai,\\
    que sempre me apoiou.}
\end{dedicatoria}

% ---------------------------------------------------------------------------------------------
% ------------------------------------------------------------------------------Agradecimentos-
% ---------------------------------------------------------------------------------------------
\begin{agradecimentos}
    Agradeço a Deus pela força e discernimento que me permitiram concluir esta etapa acadêmica. Agradeço a minha namorada por me acompanhar nessa jornada, sempre ao meu lado. Agradeço aos meus pais, que me apoiaram durante esta jornada acadêmica. Ao professor Victor de Almeida Thomaz, minha gratidão pela orientação e compartilhamento de conhecimentos essenciais para este trabalho.
\end{agradecimentos}

% ---------------------------------------------------------------------------------------------
% ------------------------------------------------------------------------------------Epígrafe-
% ---------------------------------------------------------------------------------------------
\begin{epigrafe}
	\vspace*{\fill}
	\begin{flushright}
	\textit{``Não vos amoldeis às estruturas deste mundo, \\
			mas transformai-vos pela renovação da mente, \\
			a fim de distinguir qual é a vontade de Deus: \\
			o que é bom, o que Lhe é agradável, o que é perfeito.\\
			(Bíblia Sagrada, Romanos 12, 2)} % Esse é um exemplo de epígrafe.
	\end{flushright}
\end{epigrafe}
% ---

% ---------------------------------------------------------------------------------------------
% -------------------------------------------------------------------------resumo em português-
% ---------------------------------------------------------------------------------------------
\begin{resumo}
	\vspace{\onelineskip}

Aplicativos de edição de imagens existem há muito tempo e, desde a sua criação, não deixam de apresentar desafios de usabilidade para usuários sem conhecimento técnico. O objetivo deste trabalho é desenvolver um aplicativo de edição de imagens simplificado e mais direto, utilizando o paradigma de programação de fluxo de dados, onde dados fluem entre nós através por arestas que estão interconectadas em um grafo direcionado, para isso, três soluções de software distintas foram criadas, a fim de comparar diferentes linguagens de programação e suas capacidades na implementação do paradigma de programação de fluxo de dados. A criação de tal aplicativo traz vantagens significativas, como a redução da curva de aprendizado para novos usuários e a possibilidade de indivíduos sem experiência técnica editarem imagens de forma eficaz. Além disso, o uso do paradigma de programação de fluxo de dados proporciona uma abordagem modular e flexível para a edição de imagens, facilitando a experimentação e criatividade dos usuários. Criando assim novas possibilidades para o campo de edição de imagens, democratizando o acesso a ferramentas avançadas.

\noindent 

\textbf{Palavras-chave}: Programação de fluxo de dados, Grafo Direcionado, Edição de imagens, Usabilidade, Praticidade.

\end{resumo}

% ---------------------------------------------------------------------------------------------
% ----------------------------------------------------------------------------resumo em inglês-
% ---------------------------------------------------------------------------------------------
\begin{resumo}[Abstract]
	\begin{otherlanguage*}{english}
    \vspace{\onelineskip}
    Image editing tools have existed for a long time and, since their inception, these software solutions have always presented usability challenges for non-technical users. The aim of this work is to develop a simplified and more straightforward image editing application, specifically designed for end users without technical expertise. As a key differentiator, the project explores the use of visual technologies, employing visual programming based on directed graphs to make the editing process more intuitive and accessible.

    \noindent
\textbf{Keywords}: Visual Programming, Image Editor, Directed Graph.
\end{otherlanguage*}
\end{resumo}